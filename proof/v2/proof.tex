%-----------------------------------------------------------------------------
%
%               Template for sigplanconf LaTeX Class
%
% Name:         
%
% Purpose:      A template for sigplanconf.cls, which is a LaTeX 2e class
%               file for SIGPLAN conference proceedings.
%
% Guide:        Refer to "Author's Guide to the ACM SIGPLAN Class,"
%               sigplanconf-guide.pdf
%
% Author:       Anthony Canino
%               SUNY Binghamton
%               acanino1@binghamton.edu
%
% Created:      24 August 2015
%-----------------------------------------------------------------------------


\documentclass[onecolumn,nocopyrightspace]{sigplanconf}

\usepackage{color}
\usepackage{comment}
\usepackage{quoting}
\usepackage{ragged2e}
\usepackage{changepage}
\usepackage[parfill]{parskip}

% The following \documentclass options may be useful:

% preprint      Remove this option only once the paper is in final form.
% 10pt          To set in 10-point type instead of 9-point.
% 11pt          To set in 11-point type instead of 9-point.
% authoryear    To obtain author/year citation style instead of numeric.

\newenvironment{proofcenter}[1][2em]
  {\begin{quoting}[leftmargin=#1,rightmargin=#1]\RaggedRight}
    {\end{quoting}}

\newcommand{\hmmax}{0}
\newcommand{\bmmax}{0} 

\definecolor{coralred}{rgb}{1.0, 0.25, 0.25}
\definecolor{bostonuniversityred}{rgb}{0.8, 0.0, 0.0}
\definecolor{cadmiumred}{rgb}{0.89, 0.0, 0.13}
\definecolor{darkcandyapplered}{rgb}{0.64, 0.0, 0.0}
\definecolor{britishracinggreen}{rgb}{0.0, 0.26, 0.15}
\definecolor{coolblack}{rgb}{0.0, 0.18, 0.39}
\definecolor{bittersweet}{rgb}{1.0, 0.44, 0.37}
\definecolor{bubblegum}{rgb}{0.99, 0.76, 0.8}
\definecolor{persianred}{rgb}{0.8, 0.2, 0.2}

\def\entcolor{darkcandyapplered}
\def\etcolor{persianred}

\long\def\dnote#1{{\small \ \ $\langle\langle\langle$\ \textcolor{britishracinggreen}{\textbf{#1 -David}}\
    $\rangle\rangle\rangle$\ \ }}
\long\def\anote#1{{\small \ \ $\langle\langle\langle$\ \textcolor{coolblack}{\textbf{#1 -Anthony}}\
    $\rangle\rangle\rangle$\ \ }}


\usepackage{times}
\usepackage{amsfonts}
\usepackage{subfigure}
\usepackage{amsmath} %xrightarrow
\usepackage{amssymb} %leadsto rightsquigarrow rhd mathbb Join Subset
\usepackage{array}
\usepackage{color}
\usepackage{pifont}
%\usepackage{MnSymbol}

%\usepackage{mathabx} % arrows
% the following are from owl directory
\usepackage{epsfig}
\usepackage{alltt}
\usepackage{latexsym}
\usepackage{mathpartir}
%\usepackage{mathrsfs} %mathscr
%\usepackage{bbm}  %mathbbmss
\usepackage{wasysym} %rhd, RHD
\usepackage{textcomp}

\usepackage{stackrel} % adds stack symbol below arrow option

\usepackage{setspace}

\usepackage{stmaryrd}
\usepackage{amsthm} %proof // llncs is not happy
\usepackage{amsbsy} % bold greek letter
\usepackage{mathtools} % newtagform
\usepackage{ifthen}
\usepackage{boxedminipage} 

\usepackage{bbold}
\usepackage{turnstile}
\usepackage{slashed}


\usepackage{url,hyperref, graphicx,listings}


%%%
%%% variant
\def\variant{\emph{v}}
\def\vplus{\texttt{+}}
\def\vminus{\texttt{-}}
\def\vnone{\epsilon}
\def\vbi{\texttt{*}}


%%%%%%%%%%%%%%%%%%%%
% general symbols

\def\lb{\langle}
\def\rb{\rangle}
\def\bottom{\bot}

\def\dot{.}
\def\bangsign{!}

\def\sp{\hspace*{6mm}}
\def\smallp{\sp}
\def\brs#1{\llbracket\, #1 \,\rrbracket}

\def\context#1#2#3#4{\brs{#1}_{#2, #3}}
\def\thiscontext{\Xi}


\def\df{\ \stackrel{\rm def}{=}\ }
\def\restrict#1#2{#1\mid_{#2}}



%%%%%%%%%%%%%% Inference Language %%%%%%%%%%%%%%%%

\def\ourlangi{$\langi{\ourlang}$}
\def\langi#1{#1^{-}}
\def\ti {\emph{t}}
\def\ei {\emph{ep}}
\def\fddeci{\langi{\mathit{F}}}
\def\mddeci{\langi{\mathit{M}}}
\newcommand{\fdlisti}{\overline{\fddeci}}
\newcommand{\mdlisti}{\overline{\mddeci}}
\newcommand{\constructormetai}{\langi{\mathit{K}}}
\newcommand{\classesi}{\langi{\mathit{C}}}

%\def\programcodei{\langi{\Psi}} 

\def\programcode{\mathit{P}} 

\newcommand{\kwthisp}{\texttt{\bf thisp}}
\newcommand{\kwthism}{\texttt{\bf thism}}


\def\plt{\mathtt{plt}}
\def\mlt{\mathtt{mlt}}


\def\delayset{\Upsilon}
\def\flowset{\Pi}

\def\open#1#2#3{#1 \Uparrow^{#2} #3}
\def\close#1#2#3{#1 \Downarrow_{#2} #3}





\def\concat{:}
\def\defassign{\mathrel{\Coloneqq}}
\def\assign{\mathrel{\coloneqq}}
\def\consassign{\mathrel{\coloneqq_{\tt K}}}

\def\fundef#1{\mathtt{#1}}

\DeclareMathAlphabet{\mathpzc}{OT1}{pzc}{m}{it}

\def\ovar{\mathcal{O}} % object context
\def\mvar{\mathcal{K}} % messaging context
\def\polycontext{\Lambda}

%%%%%%%%%%%%%%%%%%%%
% rule titles
\def\rtitle#1{{\sf (\textrm{R-#1})}}
\def\sbtitle#1{{\sf (\textrm{S-#1})}}
\def\esbtitle#1{{\sf (\textrm{ES-#1})}}
\def\eetitle#1{{\sf (\textrm{EE-#1})}}
\def\stitle#1{{\sf (\textrm{T-#1})}}
\def\eqtitle#1{{\sf (\textrm{EQ-#1})}}
\def\ocltitle#1{{\sf (\textrm{C-#1})}}
\def\deftitle#1{{\sf (\textrm{D-#1})}}
\def\alphatitle#1{{\sf (\textrm{CEquiv-#1})}}
\def\wftitle#1{{\sf (\textrm{WF-#1})}}
\def\opentitle#1{{\sf (\textrm{OP-#1})}}
\def\closetitle#1{{\sf (\textrm{CL-#1})}}
\def\functitle#1{{\sf (\textrm{F-#1})}}

\def\mttitle#1{{\sf (\textrm{MT-#1})}}
\def\fdtitle#1{{\sf (\textrm{FD-#1})}}
\def\mbtitle#1{{\sf (\textrm{MB-#1})}}
\def\abtitle#1{{\sf (\textrm{AB-#1})}}

\def\sititle#1{{\sf (\textrm{TI-#1})}}
\def\sbititle#1{{\sf (\textrm{SI-#1})}}

\def\ctitle#1{{\sf (\textrm{C-#1})}} 

% Titles for inline-text (t)

\def\stitlet#1{{\sf \textrm{T-#1}}}
\def\sbtitlet#1{{\sf (\textrm{S-#1})}}
\def\wftitlet#1{{\sf \textrm{WF-#1}}}
\def\mttitlet#1{{\sf \textrm{MT-#1}}}
\def\fdtitlet#1{{\sf \textrm{FD-#1}}}
\def\mbtitlet#1{{\sf \textrm{MB-#1}}}
\def\abtitlet#1{{\sf \textrm{AB-#1}}}
\def\rtitlet#1{{\sf \textrm{R-#1}}}


\def\sconjudge{\vdash_{\tt scon}}
\def\wconjudge{\vdash_{\tt wcon}}

\def\judgei{\vdash^{\tt i}}
\def\evdash{\vdash_{\tt e}}

\def\update#1#2{#1 \Leftarrow #2}

\def\dmarkers{\mathbb{DELAY}}




% math definitions

\def\subst#1#2{\{#2/#1\}}
\def\ssubst#1#2{\{#2\! \sslash \! #1\}}
\def\substsingle#1{\{#1\}}

\newcommand{\mapupdate}[1]{[ #1 ]}
\def\mapmerge{\uplus}
\def\emptymap{\emptyset_M}

\def\dom{\fundef{dom}}
\def\ran{\fundef{range}}
\def\whole{\fundef{whole}}
\def\emptyseq{\epsilon}

\def\cuptrans{\cup_+}

\def\mapmerge{\uplus}
\def\maprightmerge{\rhd}
\def\mapsub#1#2#3{#1 \mid^{#2}_{#3}}
\def\mapminus{\backslash}
\def\mapcompose{\circ}
\def\mapindomainupdate{\unrhd} % used for field update

\def\flowto{\twoheadrightarrow}

\def\ocxt{\mathcal{L}} 
\def\ccxt{\mathcal{Q}} 


\def\oitem{\ell} 
\def\citem{\mathit{q}} 
\def\instlabel{\textsf{I}} 
\def\declabel{\textsf{S}} 
\def\invlabel{\textsf{Q}} 

\def\rlabel{\textsf{r}} 



\def\ocxtvar{\textsf{ov}} 
%\def\ccxtvar{\textsf{qv}} 
\def\invlabelvar{\textsf{qv}} 
\def\modevar{\textsf{mv}} 
\def\phasevar{\textsf{pv}} 
\def\ptvar{\mathtt{pt}}
\def\mtsvar{\mathtt{mts}}
\def\mtdvar{\mathtt{mtd}}
%\def\ocpair{\emph{kv}} 

\def\phaseq{\cong}

%\def\plist#1#2{\lb\overline{#1},\overline{#2}\rb}

\def\pcmeta{\textit{T}}
\def\pcstruct#1#2{#1#2}

\def\inst#1#2#3{#1 \vdash_{\tt inst} #2 \downarrow #3}
\def\insti#1#2#3{#1 \judgei_{\tt inst} #2 \downarrow #3}

\def\mcases{\pi}
\def\vmode{\rotatebox[origin=c]{30}{$\mode$}}
\def\unvmode{\mathtt{mu}}
\def\phase{\phi}
\def\mppair{\pi}
\def\phname{\mathtt{p}}
\def\movar{\mathit{m}}
\def\phvar{\mathit{p}}

\def\phset{\Theta}
\def\moset{\Omega}


\def\modef{\mathit{MO}}
\def\phdef{\mathit{PO}}
\def\rodef{\mathit{R}}

\def\tvar{\mathit{t}}
\def\ocxtname{\mathtt{L}}

\def\pt#1{\texttt{\bf pVar}(#1)}
\def\mt#1{\texttt{\bf mVar}(#1)}

\def\lazy#1{\circlearrowleft#1}

\newcommand{\classes}{\mathit{C}}
\def\ctlist{\mathcal{C}} 
\def\CS{\mathit{Cs}} 
\def\mtlist{\mathcal{M}} 
\def\MS{\mathit{Ms}} 


\newcommand{\superset}{\mathcal{U}}

\newcommand{\constructormeta}{\mathit{K}}

\def\fddec{\mathit{F}}
\def\mddec{\mathit{M}}
\newcommand{\fdlist}{\overline{\fddec}}
\newcommand{\fdlistp}{\overline{\fddec'}}
\newcommand{\mdlist}{\overline{\mddec}}

\def\Fname{\mathtt{fd}}
\def\Mname{\mathtt{md}}
\def\Oname{\mathtt{op}}
\def\attributor{\mathit{A}}
\def\reconstructor{\mathit{R}}

\def\VAR{\mathtt{x}}
\def\YVAR{\mathtt{y}}
\def\SVAR{\mathtt{s}}

\def\Cname{\mathtt{c}}
\def\Dname{\mathtt{c}'}

% Ent Syntax

\def\Fmodes{\fundef{modes}}
\def\Feparam{\fundef{eparam}}
\def\Fethis{\fundef{ethis}}

\def\Fiparam{\fundef{iparam}}
\def\Fithis{\fundef{ithis}}

\def\Feargs{\fundef{eargs}}
\def\Fiargs{\fundef{ithis}}

\def\Fcons{\fundef{cons}}

\def\modevt{\mathtt{modev}}

\def\Fboot{\fundef{boot}}

\def\Fabody{\fundef{abody}}

\def\programcode{\mathit{P}} 

\def\listi{\iota}
\def\vlisti{\rotatebox[origin=c]{30}{$\listi$}}
\def\listd{\iota}

\def\espec{\Omega}  % energy specification
\def\tspec{\Delta}  % total energy specification

\def\econs{\omega}


\def\Checkname{\texttt{\bf check}}
\def\Mcasename{\texttt{\bf mcase}}
\def\Objname{\mathtt{obj}}

\def\mode{\mu}
\def\dynmode{\mathtt{?}}
\def\basemode{\eta}

\def\mtvar{\mathtt{mt}}
\def\moname{\mathtt{m}}
\def\fresh{\mathit{fresh}}
\def\bt{\mathit{T}}

\def\Cname{\mathtt{c}}
\def\Kappa{\mathtt{K}}
\def\cset{\Kappa}

\newcommand{\kwsnapshot}{\texttt{\bf snapshot}}

\def\classiota{\Cname\lb\listi\rb}
\def\classiotap{\Cname\lb\listi'\rb}
\def\classdyn{\Cname\lb\dynmode,\listi\rb}
\def\classargs#1{\Cname\lb#1\rb}
\def\classargsp#1{\Cname'\lb#1\rb}

\def\closure#1#2#3{\Objname(#1,#2,#3)}
\def\letx#1#2{\kwlet\ \VAR = #1\ \kwin\ #2}

\def\blame{\mathtt{p}}
\def\blamable{\mathit{b}}
\def\posblame{\blame}
\def\negblame{\overline{\blame}}

\def\snapshot#1#2#3{\kwsnapshot\ #1\ [#2,#3]}

\def\check#1#2#3{\Checkname(#1,#2,#3)}

\def\new#1{\kwnew \ #1}
\def\mcase#1#2{\{\overline{\moname : #2}\}^{#1}}
\def\mcasetag#1#2{#1 \ \triangleright \ #2}
\def\mcaseenv{\mathbf{D}}

\def\texist#1{\exists#1}
\def\tmcase#1{\Mcasename\lb#1\rb}

\def\cin{\in_{c}}

\def\ftv#1{\texttt{FTV}(#1)}

\def\judgewft{\vdash_{\tt wft}}
\def\judgewfe{\vdash_{\tt wfe}}

\def\msub{\leq}
%\def\msub{\tsub}

\def\mtitle#1{{\sf (\textrm{M-#1})}}

\def\mosrc{\mathit{Ms}}
\def\phsrc{\mathit{Ps}}
\newcommand{\st}{\mathit{T}}


\newcommand{\denergy}{\moname_0}



\def\realt#1#2{#1[#2]}

\def\Fmtype{\fundef{mtype}}
\def\Fmbody{\fundef{mbody}}
\def\Fconstructor{\fundef{constructorOK}}
\def\Fsupers{\fundef{supers}} %EZGI DEFINED function supers
\def\Fclass{\fundef{class}}
\def\Fmode{\fundef{mode}}
\def\Femode{\fundef{emode}}
\def\Fattr{\fundef{attr}}
\def\FCname{\fundef{name}}
\def\Fphase{\fundef{phase}}
\def\Fid{\fundef{ocxt}}
\def\FCmode{\fundef{mode}}
\def\FCphase{\fundef{phase}}
\def\FCccxt{\fundef{ccxt}}
\def\FCclass{\fundef{class}}
\def\FCmethod{\fundef{method}}
\def\Fupper{\fundef{upper}}
\def\Flower{\fundef{lower}}
\def\Fblamable{\fundef{blamable}}
\def\Fblame{\fundef{blame}}
\def\Finit{\fundef{init}}

\def\nodecl{\epsilon}

\newcommand{\kwany}{\texttt{\bf any}}
\def\waterfall#1#2{#1 \searrow #2}
\def\unify#1#2{#1 \leftrightarrow #2}


\def\Fupdate{\texttt{updatetypes}}
\def\Ffields{\fundef{fields}}
\def\Foverride{\fundef{override}}
\def\FconstructorOK{\fundef{constructorOK}}
\def\FFTV{\texttt{FTV}}
\def\Flabels{\texttt{labels}}
\def\Flabelsin{\texttt{labelsin}}
\def\Fvars{\texttt{vars}}
\def\Fpoly{\texttt{poly}}
\def\Fpolyset{\texttt{polyset}}
\def\Fmpath{\texttt{mpath}}
\def\Nobject{\texttt{Object}}
\def\labelsubset{\omega}

\def\kwwhere{\textrm{where}}
\def\kwdistinct{\textrm{distinct}}
\def\kwundef{\textrm{undefined}}
\def\kwifin{\textbf{mswitch}}
\def\kwphase{\textbf{phase}}
\def\kwmode{\textbf{mode}}

\def\Dpath{\underset{path}{\circlearrowleft}}
\def\Dpaths{\underset{paths}{\circlearrowleft}} %for constraint set
\def\Dpathc{\underset{pathc}{\circlearrowleft}} %for each constraint

\def\Dclass{\underset{class}{\circlearrowleft}}
\def\Dmode{\underset{mode}{\circlearrowleft}}
\def\Dphase{\underset{phase}{\circlearrowleft}}

\def\instmode{\mathsf{InstModes}}
\def\instphase{\mathsf{InstPhases}}

\def\labelobj#1#2{<{#1} ,{#2} >}

%\def\Lname{\mathsf{L}}
\def\Lname{\eta}
\def\newvar{\rho}
\def\msgvar{\varrho}




\def\tenvname{\mathsf{t}}
\def\invnote{\mathsf{*}}

\def\calls{\Delta}
\def\callsitem{\delta}

\def\dcalls{\Delta}
\def\dcallsitem{\delta}

\def\zip{\longmapsto}

%\newcommand{\program}{\mathit{P}}


%Coqa type system syntax
\newcommand{\kwmain} {{\bf main}}
\def\tsub{\mathbin{<\vcentcolon}}

\def\ntsub{\mathbin{\nless\vcentcolon}}

\def\tswitch{\mathbin{\ll\vcentcolon}}
\def\adaptto{\ll}
\def\adaptcxt{\Upsilon}


\def\cons{\kappa} 
\def\t{\tau}
\def\kwlobjt#1#2{{ #2@ #1}}


\def\econtextletter{\mathcal{X}}

% closure related
\def\cvar{\beta}
\def\cvarlist{\mathcal{B}}
\def\cvarext{\Hat{\beta}}
\def\ccons{\omega}

\def\sigvar{\alpha}
\def\sigvarlist{\mathcal{A}}
\def\sigvarext{\Hat{\sigvar}}
\def\sigvarsuperlist{\mathcal{S}}

\def\pclosure#1#2#3{\ctlist \stackrel{#1}{\looparrowright}#3} 

\def\porder{\mathrel{<_{cpu}}}
\def\morder{\mathrel{<\vcentcolon}}


\def\labelvar{\ell}



\def\topclred{\hookrightarrow}

\def\clred{\stackrel{\calls}{\hookrightarrow}}
\def\clredp#1{\stackrel{#1}{\hookrightarrow}}

\def\ocldash{\vdash_\mathrm{c}}
\def\judgeprogram{\vdash_\mathrm{p}}
\def\judgeclass{\vdash_\mathrm{cls}}
\def\judgemethod{\vdash_\mathrm{m}}
\def\judgefield{\vdash_\mathrm{f}}
\def\judgeh{\vdash_\mathrm{h}}
\def\judgecell{\vdash_\mathrm{hc}}
\def\judger{\vdash_\mathrm{r}}
\def\judgesubst{\vdash_\mathrm{WF}}

\newcommand{\kwcxt}{\texttt{\bf cxt}} % current context
\newcommand{\kwlmain}{\texttt{\bf main}}

\newcommand{\kwtowner}{\mathtt{thost}}
\newcommand{\kwtmain}{\mathtt{tmain}}
\newcommand{\cconsboot}{\mathtt{boot}}
\newcommand{\scale}[2]{ \looparrowright #2}

\newcommand{\kwscale}{\texttt{\bf scale}}



%%%%%%%%%%%%%%%%%%%%
%Coqa dynamics syntax

\def\dset{\Sigma}

\def\sreduct{\Longrightarrow}
\def\initreduct{\Longrightarrow_{\tt init}}
\def\sreductp#1{\stackrel{#1}{\Longrightarrow}}


\def\cloreduct#1{\stackrel{#1}{\Longrightarrow}}

\def\effcloreduct#1#2{\stackrel{#1}{\Rrightarrow}_{#2}}

%\def\cyclevar{\Xi}

\def\nd{\updownarrow}

\def\bleft{\Yleft}
\def\bright{\Yright}



\newcommand{\RID}{\mathit{o}}
\newcommand{\val}{\mathit{v}}
\newcommand{\obj}{\mathit{o}}
\newcommand{\redreal}[2]{{\bf E}_{#2}[\, #1 \,]}

\newcommand{\mcasered}[2]{{\bf D}_{#2}[\, #1 \,]}

\newcommand{\sred}[1]{{\bf Es}[\, #1 \,]}
\def\env{\mathbf{E}}
\def\heap{\mathbf{H}}

\def\renv{\mathbf{R}}
\def\renvm{\mathbf{Rm}}
\def\renvp{\mathbf{Rp}}
\def\renvf{\mathbf{Rf}}


\def\full{\mathcal{LF}}
\def\partial{\mathcal{LU}}




\def\inO#1#2{\lb #2 \rb^{#1}}
\def\inT#1#2{\brs{#2}^{#1}}
\def\accessop#1#2{\ndtstile{#1}{ #2}}

%\def\env#1#2{\mathbf{E}_{#1}[\, #2 \,]}

%\def\kwin#1#2#3{\texttt{\bf in}(#1, #2, #3)}

\def\inv#1#2#3#4{#1\!:: \!#2 \!: \! #3(#4)}
\def\hcell#1#2#3#4{ \lb    #3;  #4 \rb}

\def\nocheckadd#1#2#3#4{ #1 +_{#4} \lb  #2;  #3 \rb}
\def\checkadd#1#2#3#4{ #1 \uplus_{#4} \lb  #2;  #3 \rb}

%\def\add#1#2#3{ #1 \uplus \lb  #2;  #3 \rb}


\def\kwldirin#1#2#3{\texttt{\bf in}(#1,  #3)}
\newcommand{\diverge}{\Uparrow}
\newcommand{\kwwait}{\texttt{\bf wait}}
\newcommand{\Fd}{\mathit{Fd}}

\newcommand{\kwpost}{\texttt{\bf post}}

\def\augG#1#2#{ #1 \lhd  #2}

\def\induces{\xrightarrow{\textsc{induces}}} 


\def\H{H}
\def\Ht{\mathcal{H}}
\def\A{\Sigma} 
\def\T{e}
\def\RIDT{p}
\def\RIDE{q}
\def\AnameT{\Aname_{\mathrm{u}}}
\def\AnameE{\Aname_{\mathrm{e}}}
\def\LnameT{\Lname_{\mathrm{u}}}
\def\LnameE{\Lname_{\mathrm{e}}}
\def\alphaT{\alpha_{\mathrm{u}}}

\def\stackinfo{\mathit{Z}}
\def\gammasub{<:}

%\def\Ht{\Gamma} 
\def\Tt{\mathcal{T}} 
\newcommand{\unfold}[1]{\fundef{signature}_{\tt #1}}

% llncs doesn't use these.
%
%\newtheorem{corollary}{Corollary}
%\newtheorem{principle}{Main Principle}
%\newtheorem{cprinciple}{Principle}

%\newtheorem{mainlemma}{Main Lemma}
%\renewcommand{\themainlemma}{M\arabic{mainlemma}}  
%\numberwithin{equation}{lemma}
%\renewcommand{\casename}{}
%\renewcommand{\theequation}{\thelemma.\casename\arabic{equation}}
%\newcommand{\consno}[3]{\mathbf{\thelemma.#1.(#2)#3}}

\def\consimp#1{\ \xrightharpoonup{#1}\ }

% subject reduction proof



\newcommand{\kwlocal}{\texttt{\bf local}}
\newcommand{\kwidle}{\texttt{\bf idle}}



\newcommand{\kwobj}{\texttt{\bf obj}}


%%%%%%%%%%%%%%%%%%%%
%Java syntax
%\newcommand{\kwattribute}{\texttt{\bf attribute}}
\newcommand{\kwreconstruct}{\texttt{\bf reconstruct}}
\newcommand{\kwusing}{\texttt{\bf using}}
\newcommand{\kwblame}{\texttt{\bf blame}}
%\newcommand{\kwas}{\texttt{\bf as}}
%\newcommand{\kwin}{\texttt{\bf in}}
%\newcommand{\kwadapt}{\texttt{\bf adapt}}
%\newcommand{\kwdistinct}{\texttt{\bf distinct}}
%\newcommand{\kwecontext}{\texttt{\bf econtext}}
\newcommand{\kwclass}{\texttt{\bf class}}
\newcommand{\kwinterface}{\texttt{\bf interface}}
\newcommand{\kwint}{\texttt{\bf int}}
\newcommand{\kwfloat}{\texttt{\bf double}}
\newcommand{\kwextends}{\texttt{\bf extends}}
\newcommand{\kwimplements}{\texttt{\bf implements}}
\newcommand{\kwthis}{\texttt{\bf this}}
\newcommand{\kwnew}{\texttt{\bf new}}
\newcommand{\kwreturn}{\texttt{\bf return}}
\newcommand{\kwswitch}{\texttt{\bf switch}}
\newcommand{\kwvoid}{\texttt{\bf void}}
\newcommand{\kwfinal}{\texttt{\bf final}}
\newcommand{\kwabstract}{\texttt{\bf abstract}}
\newcommand{\kwpublic}{\texttt{\bf public}}
\newcommand{\kwprotected}{\texttt{\bf protected}}
\newcommand{\kwprivate}{\texttt{\bf private}}
\newcommand{\kwboolean}{\texttt{\bf boolean}}
\newcommand{\kwsuper}{\texttt{\bf super}}
\newcommand{\kwfor}{\texttt{\bf for}}
\newcommand{\kwwhile}{\texttt{\bf while}}
\newcommand{\kwchar}{\texttt{\bf char}}
\newcommand{\kwinner}{\texttt{\bf inner}}
\newcommand{\kwnull}{\texttt{\bf null}}
\newcommand{\kwif}{\texttt{\bf if}}
\newcommand{\kwand}{\texttt{\bf and}}
\newcommand{\kwor}{\texttt{\bf or}}
\newcommand{\kwelse}{\texttt{\bf else}}
\newcommand{\kwbyte}{\texttt{\bf byte}}
\newcommand{\kwword}{\texttt{\bf word}}
\newcommand{\kwtrue}{\texttt{\bf true}}
\newcommand{\kwfalse}{\texttt{\bf false}}
\newcommand{\kwsynchronized} {\texttt{\bf synchronized}}
\newcommand{\kwstatic}{\texttt{\bf static}}
\newcommand{\kwthrow}{\texttt{\bf throw}}
\newcommand{\kwexception}{\texttt{\bf exception}}
\newcommand{\kwcase}{\texttt{\bf case}}

\newcommand{\kwphases}{\texttt{\bf phases}}
\newcommand{\kwmodes}{\texttt{\bf modes}}

\newcommand{\kwin}{\texttt{\bf in}}
\newcommand{\kwlet}{\texttt{\bf let}}

%%%%%%%%%%%%%%%%%%%%
% boxes 

\def\nocaptionrule{\sp} % to make llncs happy


\newcommand{\codebox}[1]{
\begin{center}
\noindent\begin{minipage}[c]{\columnwidth} 
\begin{center}
\vskip 2ex
\begin{tabular*}{\columnwidth}{c}\end{tabular*}
\vskip 1ex
$
\begin{array}{l}
#1
\end{array}
$ 
\vskip 1ex
%\begin{tabular*}{\columnwidth}{c}\hline\end{tabular*}
%\vskip 2ex
\end{center}
\end{minipage} 
\end{center}
\noindent
}


\newcommand{\figurebox}[1]{
\begin{center}
\noindent\begin{minipage}[c]{\columnwidth} 
\begin{center}
\vskip 0ex
\begin{tabular*}{\columnwidth}{c}\end{tabular*}
\begin{tabular*}{\columnwidth}{c}\hline\end{tabular*}
\vskip 1ex
#1
\vskip 1ex
\begin{tabular*}{\columnwidth}{c}\hline\end{tabular*}
\vskip 2ex
\end{center}
\end{minipage} 
\end{center}

}

\newcommand{\twofigurebox}[1]{
\begin{center}
\noindent\begin{minipage}[c]{\textwidth} 
\begin{center}
\vskip 2ex
\begin{tabular*}{\textwidth}{c}\end{tabular*}
\begin{tabular*}{\columnwidth}{c}\hline\end{tabular*}
\vskip 1ex
#1
\vskip 1ex
%\begin{tabular*}{\columnwidth}{c}\hline\end{tabular*}
%\vskip 2ex
\end{center}
\end{minipage} 
\end{center}

}



\newcommand{\figureboxtwocol}[1]{
\begin{center}
\noindent\begin{minipage}[c]{\textwidth} 
\begin{center}
\vskip 2ex
\begin{tabular*}{\textwidth}{c}\end{tabular*}
\vskip 1ex
#1
\vskip 1ex
%\begin{tabular*}{\columnwidth}{c}\hline\end{tabular*}
%\vskip 2ex
\end{center}
\end{minipage} 
\end{center}

}

%%%%%%%%%%%%%%%%%%%%
% comments



%\long\def\snote#1{{\small \textbf{\textit{(#1 -- SS)}}}}
%\long\def\dnote#1{{\small \ \ $\langle\langle\langle$\ \textbf{#1 -YL}\
%    $\rangle\rangle\rangle$\ \ }}

% non-silenceable notes
%\long\def\newsnote#1{{\small \textbf{\textit{(#1 -- SS)}}}}
%\long\def\newdnote#1{{\small \ \ $\langle\langle\langle$\ \textbf{#1 -YL}\
%    $\rangle\rangle\rangle$\ \ }}


\long\def\omitthis#1{}
\long\def\infloat#1{#1}

\def\nonotes{ % silence notes
\long\def\snote{\omitthis}
\long\def\dnote{\omitthis}
}


\long\def\inlong#1{#1}
\long\def\inshort#1{}
\long\def\inproof#1{}


%\definecolor{mygray}{gray}{0.85}
\definecolor{mygray}{cmyk}{0.1,0.3,0.1,0.50}





\usepackage{listings}
\usepackage{fixltx2e}
\usepackage{tikz}
\usepackage{titlesec}
\usepackage{tabularx}
\usepackage{balance}
\usepackage{float}
\usepackage{graphicx}
\usepackage{moresize}
\usepackage[inline,shortlabels]{enumitem}
\usepackage[font={normalsize},skip=1pt]{caption}

\renewcommand{\figurename}{Fig.}


\newcolumntype{Y}{>{\centering\arraybackslash}X}

\newtheorem{theorem}{Theorem} %[section]
\newtheorem{lemma}{Lemma}
\newtheorem{corollary}{Corollary}
\newtheorem{definition}{Definition}
\theoremstyle{lessintrusive} 
\newtheorem{question}{Research Question}
\theoremstyle{plain}

\newtheoremstyle{custom}
{5pt}   % Space above
{5pt}   % Space below
{}          % Body font
{0pt}       % Indent amount (empty value is the same as 0pt)
{\itshape}  % Theorem head font
{}          % Punctuation after theorem head
{5pt plus 1pt minus 1pt} % Space after theorem head
{\thmname{#1} \thmnote{#3}} % Theorem head spec

\theoremstyle{custom}
\newtheorem*{case}{Case}

\newtheoremstyle{subcase-custom}
{3pt}   % Space above
{3pt}   % Space below
{}          % Body font
{0pt}       % Indent amount (empty value is the same as 0pt)
{\itshape}  % Theorem head font
{}          % Punctuation after theorem head
{5pt plus 1pt minus 1pt} % Space after theorem head
{\thmname{#1} \thmnote{#3}} % Theorem head spec

\theoremstyle{subcase-custom}
\newtheorem*{subcase}{Subcase}

\clubpenalty = 10000
\widowpenalty = 10000
\displaywidowpenalty = 10000 

\newenvironment{subcase-env}
{
  \begin{adjustwidth}{2em}{2em}
}
{
  \end{adjustwidth}
}

\newcommand{\ourlang}{\textsc{Ent}}

\def\econsexp#1#2#3{#1\msub#2\msub#3}
\def\econsset#1#2#3{\{#1\msub#2,#2\msub#3,#1\msub#3\}}

\begin{document}

\title{Proactive and Adaptive Energy-Aware Programming with Hybrid Typing --- Proofs \vspace{-2em}}

\authorinfo{}{}{}

\maketitle

\section{Proofs} 

\anote{Internal vs External issue regarding : wft, abody, preservation.}

% LEMMA : Weakening
\begin{lemma}[Weakening]
\label{pf:weakening}
\leavevmode
\begin{enumerate}[(\arabic*)] 

\item If $\cset\judgewft\t$ and $\cset \models \{\basemode\msub\basemode'\}$ then $\cset,\basemode\msub\basemode'\judgewft\t$.

\item If $\cset\vdash\t\tsub\t'$ and $\cset \models \{\basemode\msub\basemode'\}$ then $\Gamma;\cset,\basemode\msub\basemode'\vdash\t\tsub\t'$.

\item If $\Gamma;\cset\vdash e:\t$, and $\cset \models \{\basemode\msub\basemode'\}$, then $\Gamma;\cset,\basemode\msub\basemode' \vdash e:\t$.

\item If $\Gamma;\cset\vdash e:\t$, and $\Gamma \vdash \YVAR:\t'$, then $\Gamma,\YVAR:\t';\cset \vdash e:\t$.

\end{enumerate}
\end{lemma}

\begin{proof}
Each is proved by straightforward induction on the derivations of $\cset\judgewft\t$, $\cset\vdash\t\tsub\t'$, and $\Gamma;\cset\vdash e:\t$.
\end{proof} 

% LEMMA : Mode substitution preserves entailment
\begin{lemma}
\label{pf:modesubstitution-preserves-entailment}
If $\cset,\basemode\msub\mtvar,\mtvar\msub\basemode'\models\{\basemode_2\msub\basemode_2'\}$, $\cset\models\{\basemode\msub\basemode'',\basemode''\msub\basemode'\}$, and $\mtvar \not\in \cset$, then $\cset\subst{\mtvar}{\basemode''}\models\{\basemode_2\subst{\mtvar}{\basemode''}\msub\basemode_2'\subst{\mtvar}{\basemode''}\}$.
\end{lemma}
%
\begin{proof}
Trivial.
\end{proof}

% LEMMA : Mode substitution preserves type well-formedness
\begin{lemma}[Mode Substitution Perserves Type Well-Formedness]
\label{pf:modesubstitution-preserves-wellformedness}
If $\cset,\basemode\msub\mtvar,\mtvar\msub\basemode' \judgewft \t$, $\cset\models\{\basemode\msub\basemode'',\basemode''\msub\basemode'\}$, and $\mtvar \not\in \cset$, then $\cset\subst{\mtvar}{\basemode''} \judgewft \t\subst{\mtvar}{\basemode''}$.
\end{lemma}
\begin{proof}
By induction on the derivation of $\cset,\basemode\msub\mtvar,\mtvar\msub\basemode' \judgewft \t$.

\begin{case}[\wftitlet{Top}]
\begin{tabular}[t]{>{$}l<{$} >{$}l<{$} >{$}l<{$}}
\bt = \Nobject\lb\basemode\rb & & \\
\end{tabular}\\ 
Trivial.
\end{case}

\begin{case}[\wftitlet{MCase}] 
\begin{tabular}[t]{>{$}l<{$} >{$}l<{$} >{$}l<{$}}
\bt = \tmcase{\bt_1} & & \\
\cset,\basemode\msub\mtvar,\mtvar\msub\basemode' \judgewft \bt_1 & & \\
\end{tabular}\\ 
By the induction hypothesis, $\cset\subst{\mtvar}{\basemode''}\judgewft\bt_1\subst{\mtvar}{\basemode''}$. Then, by WF-MCase, $\cset\subst{\mtvar}{\basemode''}\judgewft \tmcase{\bt_1\subst{\mtvar}{\basemode''}}$.
\end{case}

\begin{case}[\wftitlet{Class}] 
\begin{tabular}[t]{>{$}l<{$} >{$}l<{$} >{$}l<{$}}
\bt = \Cname\lb\overline{\basemode}\rb & & \\
\kwclass\ \Cname\ \tspec\ \kwextends\ \Dname\ \dots \in \programcode & \Feparam(\tspec) = \listi' & \Fcons(\tspec) = \cset' \\
\cset,\basemode\msub\mtvar,\mtvar\msub\basemode'\models\cset'\subst{\listi'}{\overline{\basemode}} & \cset,\basemode\msub\mtvar,\mtvar\msub\basemode'\judgewft\Dname\lb\overline{\basemode}\rb & \\ 
\end{tabular}\\
By the induction hypothesis, $\cset\subst{\mtvar}{\basemode''}\judgewft\Dname\lb\overline{\basemode}\rb\subst{\mtvar}{\basemode''}$. Lemma \ref{pf:modesubstitution-preserves-entailment} gives us $\cset_1,\cset_2\subst{\mtvar}{\basemode''}\models\cset'\subst{\listi'}{\listi}\subst{\mtvar}{\basemode''}$.

Then, by \wftitlet{Class}, $\cset\subst{\mtvar}{\basemode''}\judgewft\Cname\lb\overline{\basemode}\rb\subst{\mtvar}{\basemode''}$.
\end{case} 

\begin{case}[\wftitlet{ClassDyn}] 
\begin{tabular}[t]{>{$}l<{$} >{$}l<{$} >{$}l<{$}}
\bt = \Cname\lb\dynmode,\overline{\basemode}\rb & & \\
\kwclass\ \Cname\ \dynmode\rightarrow\econs,\espec\ \kwextends\ \Dname\ \dots \in \programcode & \Feparam(\dynmode\rightarrow\econs,\espec) = \listi' & \Fcons(\dynmode\rightarrow\econs,\espec) = \cset' \\
\cset,\basemode\msub\mtvar,\mtvar\msub\basemode'\models\cset'\subst{\listi'}{\overline{\basemode}} & \cset,\basemode\msub\mtvar,\mtvar\msub\basemode'\judgewft\Dname\lb\dynmode,\overline{\basemode}\rb & \\ 
\end{tabular}\\
Similar.
\end{case} 
\end{proof}

% LEMMA : Mode substitution preserves subtyping
\begin{lemma}[Mode Substitution Perserves Subtyping]
\label{pf:modesubstitution-preserves-subtyping}
If $\cset,\basemode\msub\mtvar,\mtvar\msub\basemode'\vdash\t\tsub\t'$, $\cset\models\{\basemode\msub\basemode'',\basemode''\msub\basemode'\}$, and $\mtvar \not\in \cset$, then $\cset\subst{\mtvar}{\basemode''}\vdash\t\subst{\mtvar}{\basemode''}\tsub\t'\subst{\mtvar}{\basemode''}$.
\end{lemma}

\begin{proof}
Induction on the derivation of $\cset,\basemode\msub\mtvar,\mtvar\msub\basemode'\vdash\t\tsub\t'$.

\begin{case}[\sbtitlet{Dynamic}] 
\begin{tabular}[t]{>{$}l<{$} >{$}l<{$} >{$}l<{$}}
\t = \Cname\lb\mode,\overline{\basemode}\rb & \t' = \Cname\lb\dynmode,\overline{\basemode}\rb & \\
\end{tabular}\\
If $\mode = \mtvar$, then we have $\cset\subst{\mtvar}{\basemode''}\vdash\Cname\lb\basemode'',\overline{\basemode}\subst{\mtvar}{\basemode''}\rb\tsub\Cname\lb\dynmode,\overline{\basemode}\subst{\mtvar}{\basemode''}\rb$. If $\mode \neq \mtvar$, then we have $\cset\subst{\mtvar}{\basemode''}\vdash\Cname\lb\mode,\overline{\basemode}\subst{\mtvar}{\basemode''}\rb\tsub\Cname\lb\dynmode,\overline{\basemode}\subst{\mtvar}{\basemode''}\rb$. Both cases are exactly what is needed.
\end{case} 

\begin{case}[\sbtitlet{Mcase}] 
\begin{tabular}[t]{>{$}l<{$} >{$}l<{$} >{$}l<{$}}
\t = \tmcase{\bt_1} & \t' = \tmcase{\bt_1'} & \\
\cset,\basemode\msub\mtvar,\mtvar\msub\basemode'\vdash\bt_1\tsub\bt_1' & & \\
\end{tabular}\\ \\
By the induction hypothesis, $\cset\subst{\mtvar}{\basemode''}\vdash\bt_1\subst{\mtvar}{\basemode''}\tsub\bt_1'\subst{\mtvar}{\basemode''}$ Then, by S-MCase, $\cset\subst{\mtvar}{\basemode''}\vdash\tmcase{\bt_1\subst{\mtvar}{\basemode''}}\tsub\tmcase{\bt_1'\subst{\mtvar}{\basemode''}}$.
\end{case} 

\begin{case}[\sbtitlet{Exists}] 
\begin{tabular}[t]{>{$}l<{$} >{$}l<{$} >{$}l<{$}}
\t = \texist{\econs}.\t_{1} & \t' = \t_{1} & \\  
\econs = \econsexp{\basemode_{1}}{\mtvar_{1}}{\basemode_{2}} & \cset_{1},\basemode\msub\mtvar,\mtvar\msub\basemode',\cset_{2} \models \{\basemode_{1}\msub\mtvar_{1},\mtvar_{1}\msub\basemode_{2}\}\cup\cset' & \mtvar_{1} \not\in \cset' \\
\end{tabular}\\ \\ 

\anote{Come back to prove.}

\end{case}

\begin{case}[\sbtitlet{Class}] 
\begin{tabular}[t]{>{$}l<{$} >{$}l<{$} >{$}l<{$}}
\t = \classiota & \t' = \Dname\lb\listi\rb & \\
\kwclass\ \Cname\ \tspec\ \kwextends\ \Dname\ \dots\ \in \programcode & \Feparam(\tspec) = \listi' & \\ 
\cset,\basemode\msub\mtvar,\mtvar\msub\basemode' = \Fcons(\tspec) & & \\
\cset,\basemode\msub\mtvar,\mtvar\msub\basemode'\judgewft\Dname\lb\listi\rb & & \\
\end{tabular}\\ \\
By Lemma \ref{pf:modesubstitution-preserves-wellformedness} we have $\cset\subst{\mtvar}{\basemode''} \judgewft\Cname\lb\listi\rb\subst{\mtvar}{\basemode''}$ and $\cset\subst{\mtvar}{\basemode''} \judgewft\Dname\lb\listi\rb\subst{\mtvar}{\basemode''}$. Lemma \ref{pf:modesubstitution-preserves-entailment} we have $\cset\subst{\mtvar}{\basemode''} \models \Fcons(\tspec)\subst{\mtvar}{\basemode''}$.

Then, by S-Class, $\cset\subst{\mtvar}{\basemode''}\vdash\Cname\lb\listi\subst{\mtvar}{\basemode''}\rb\tsub\Dname\lb\listi\subst{\mtvar}{\basemode''}\rb$. 

\end{case}

\end{proof}

% LEMMA : a substitution on mtype of type t gives a substiution over the resultant types of mtype
\begin{lemma}
\label{pf:fmtype-substitutes}
If $\cset,\basemode\msub\mtvar,\mtvar\msub\basemode'\judgewft\classiota$, $\cset\models\{\basemode\msub\basemode'',\basemode''\msub\basemode'\}$, $\mtvar \not\in \cset$, and $\Fmtype(\Mname,\classiota) = \overline{\bt}\rightarrow\bt$ then $\Fmtype(\Mname,\classiota\subst{\mtvar}{\basemode''}) = \overline{\bt\subst{\mtvar}{\basemode''}}\rightarrow\bt'\subst{\mtvar}{\basemode''}$. 
\end{lemma} 

\begin{proof}
Induction on the derivation of $\Fmtype(\Mname,\classiota) = \overline{\bt}\rightarrow\bt$.

\begin{case}[\mttitlet{Class}] 
\begin{tabular}[t]{>{$}l<{$} >{$}l<{$} >{$}l<{$}}
\end{tabular}
\end{case}

\begin{case}[\mttitlet{Super}] 
\begin{tabular}[t]{>{$}l<{$} >{$}l<{$} >{$}l<{$}}
\end{tabular}
\end{case}

\end{proof}

% LEMMA : a substitution on fields of type t gives a substiution over the resultant types of fields
\begin{lemma}
\label{pf:ffields-substitutes}
If $\cset,\basemode\msub\mtvar,\mtvar\msub\basemode'\judgewft\classiota$, $\cset\models\{\basemode\msub\basemode'',\basemode''\msub\basemode'\}$, $\mtvar \not\in \cset$ and $\Ffields(\bt) = \overline{\bt}\ \overline{\Fname}$ then $\Ffields(\classiota\subst{\mtvar}{\basemode''}) = \overline{\bt\subst{\mtvar}{\basemode''}}\ \overline{\Fname}$.
\end{lemma} 

\begin{proof}
Induction on the derivation of $\Ffields(\Mname,\classiota) = \overline{\bt}\ \overline{\Fname}$.

\begin{case}[\fdtitlet{Class}] 
\begin{tabular}[t]{>{$}l<{$} >{$}l<{$} >{$}l<{$}}
\end{tabular}
\end{case}

\begin{case}[\fdtitlet{Object}] 
\begin{tabular}[t]{>{$}l<{$} >{$}l<{$} >{$}l<{$}}
\end{tabular}
\end{case}

\end{proof}

% LEMMA : Mode substitution preserves typing
\begin{lemma}[Mode Substitution Preserves Typing]
\label{pf:modesubstitution-preserves-typing}
If $\Gamma;\cset,\basemode\msub\mtvar,\mtvar\msub\basemode'\vdash e:\t$, $\cset\models\{\basemode\msub\basemode'',\basemode''\msub\basemode'\}$, and $\mtvar \not\in \cset$, then $\Gamma\subst{\mtvar}{\basemode''};\cset\subst{\mtvar}{\basemode''}\vdash e\subst{\mtvar}{\basemode''}:\t\subst{\mtvar}{\basemode''}$.
\end{lemma}

\begin{proof}
Induction on the derivation of $\Gamma;\cset,\basemode\msub\mtvar,\mtvar\msub\basemode'\vdash e:\t$.

\begin{case}[\stitlet{Var}] 
\begin{tabular}[t]{>{$}l<{$} >{$}l<{$} >{$}l<{$}}
e = \VAR & \t = \Gamma(\VAR) & \\
\end{tabular}\\
\anote{Our substitution does not effect types directly; it acts on thier parameteres. I think I need a subcase analysis here.}
\end{case}

\begin{case}[\stitlet{New}] 
\begin{tabular}[t]{>{$}l<{$} >{$}l<{$} >{$}l<{$}}
e = \new{\classiota} & \t = \classiota \\
\listi = \dynmode,\listi'\ \textrm{iff}\ \kwclass\ \Cname\ \tspec \dots \in \programcode\ \textrm{and}\ \Fethis(\tspec) = \dynmode & & \\
\listi \neq \dynmode, \listi'\ \textrm{iff}\ \kwclass\ \Cname\ \tspec \dots \in \programcode\ \textrm{and}\ \Fethis(\tspec) \neq \dynmode & & \\
\cset,\basemode\msub\mtvar,\mtvar\msub\basemode' \models \Fcons(\tspec) & \cset,\basemode\msub\mtvar,\mtvar\msub\basemode' \judgewft \classiota & \\
\end{tabular}\\ \\
Using Lemmas \ref{pf:modesubstitution-preserves-entailment} and \ref{pf:modesubstitution-preserves-wellformedness} gives us $\cset\subst{\mtvar}{\basemode''} \models \Fcons(\tspec)\subst{\mtvar}{\basemode''}$ and $\cset\subst{\mtvar}{\basemode''} \vdash \classiota\subst{\mtvar}{\basemode''}$. Then, by \stitlet{New}, we have $\Gamma\subst{\mtvar}{\basemode''};\cset\subst{\mtvar}{\basemode''}\vdash \new{\classiota}\subst{\mtvar}{\basemode''}:\classiota\subst{\mtvar}{\basemode''}$.

\end{case}

\begin{case}[\stitlet{Cast}] 
\begin{tabular}[t]{>{$}l<{$} >{$}l<{$} >{$}l<{$}}
e = (\bt)e_1 & \t = \bt & \\
\Gamma;\cset,\basemode\msub\mtvar,\mtvar\msub\basemode' \vdash e_1:\bt' & & \\
\end{tabular}\\
Easy.
\end{case} 

\begin{case}[\stitlet{Msg}] 
\begin{tabular}[t]{>{$}l<{$} >{$}l<{$} >{$}l<{$}}
e = e_1.\Mname(\overline{e_1}) & \t = \bt & \\ 
\Gamma;\cset,\basemode\msub\mtvar,\mtvar\msub\basemode'\vdash e_1:\classiota & \Gamma;\cset,\basemode\msub\mtvar,\mtvar\msub\basemode'\vdash\overline{e_1}:\overline{\bt} & \\
\Fmtype(\Mname,\classiota) = \overline{\bt}\rightarrow\bt & \Gamma;\cset,\basemode\msub\mtvar,\mtvar\msub\basemode'\vdash\kwthis:\bt_{this} & \\
\cset,\basemode\msub\mtvar,\mtvar\msub\basemode'\models \{\Fmode(\classiota)\msub\Fmode(\bt_{this})\} & \Fmode(\classiota) \neq\ \dynmode & \\
\end{tabular}\\ \\
By the induction hypothesis we have,
\begin{proofcenter}
$\Gamma\subst{\mtvar}{\basemode''};\cset\subst{\mtvar}{\basemode''}\vdash e_1\subst{\mtvar}{\basemode''}:\classiota\subst{\mtvar}{\basemode''}$ \\
$\Gamma\subst{\mtvar}{\basemode''};\cset\subst{\mtvar}{\basemode''}\vdash\overline{e_1\subst{\mtvar}{\basemode''}}:\overline{\bt\subst{\mtvar}{\basemode''}}$ \\
$\Gamma\subst{\mtvar}{\basemode''};\cset\subst{\mtvar}{\basemode''}\vdash\kwthis\subst{\mtvar}{\basemode''}:\bt_{this}\subst{\mtvar}{\basemode''}$. \\
\end{proofcenter}

Now, by Lemma \ref{pf:fmtype-substitutes} we have $\Fmtype(\Mname,\classiota\subst{\mtvar}{\basemode''}) = \overline{\bt\subst{\mtvar}{\basemode''}}\rightarrow\bt\subst{\mtvar}{\basemode''}$. 

Using Lemma \ref{pf:modesubstitution-preserves-entailment} gives us $\cset\subst{\mtvar}{\basemode''}\models\{\Fmode(\classiota\subst{\mtvar}{\basemode''})\msub\Fmode(\bt_{this}\subst{\mtvar}{\basemode''})\}$.

Then, by T-Msg, we have $\Gamma\subst{\mtvar}{\basemode''};\cset\subst{\mtvar}{\basemode''}\vdash e_1\subst{\mtvar}{\basemode''}.\Mname(\overline{e_1\subst{\mtvar}{\basemode''}}):\bt\subst{\mtvar}{\basemode''}$.

\end{case}

\begin{case}[\stitlet{Field}] 
\begin{tabular}[t]{>{$}l<{$} >{$}l<{$} >{$}l<{$}}
e = e_1.\Fname_i & \t = \bt_i & \\
\Gamma;\cset,\basemode\msub\mtvar,\mtvar\msub\basemode'\vdash e_1:\classiota & \Ffields(\classiota) = \overline{\bt}\ \overline{\Fname} & \\
\Gamma;\cset,\basemode\msub\mtvar,\mtvar\msub\basemode'\vdash \kwthis:\bt_{this} & \cset\basemode\msub\mtvar,\mtvar\msub\basemode' \models \{\Fmode(\classiota)\msub\Fmode(\bt_{this})\} & \\ 
\Fmode(\classiota) \neq \dynmode & & \\
\end{tabular}\\ \\
By the induction hypothesis we have,
\begin{proofcenter}
$\Gamma\subst{\mtvar}{\basemode''};\cset\subst{\mtvar}{\basemode''}\vdash e_1\subst{\mtvar}{\basemode}:\classiota\subst{\mtvar}{\basemode''}$ \\
$\Gamma\subst{\mtvar}{\basemode''};\cset\subst{\mtvar}{\basemode''}\vdash \kwthis:\bt_{this}\subst{\mtvar}{\basemode}$.\\
\end{proofcenter}

Now by Lemma \ref{pf:ffields-substitutes} we have $\Ffields(\classiota\subst{\mtvar}{\basemode''}) = \overline{\bt\subst{\mtvar}{\basemode''}}\ \overline{\Fname}$.

Using Lemma \ref{pf:modesubstitution-preserves-entailment} gives us $\cset\subst{\mtvar}{\basemode''}\models\{\Fmode(\classiota\subst{\mtvar}{\basemode''})\msub\Fmode(\bt_{this}\subst{\mtvar}{\basemode''})\}$.

Then, by T-Field, we have $\Gamma\subst{\mtvar}{\basemode''};\cset\subst{\mtvar}{\basemode''}\vdash e_1\subst{\mtvar}{\basemode''}.\Fname_i:\bt_i\subst{\mtvar}{\basemode''}$.

\end{case}

\begin{case}[\stitlet{Snapshot}] 
\begin{tabular}[t]{>{$}l<{$} >{$}l<{$} >{$}l<{$}}
e = \snapshot{e_1}{\basemode_1}{\basemode_2} & \t = \texist{\econs}.\Cname\lb\mtvar_1,\listi\rb & \\
\Gamma;\cset,\basemode\msub\mtvar,\mtvar\msub\basemode' \vdash e_1:\classdyn & \econs = \basemode_1\msub\mtvar_1\msub\basemode_2 & \\
\end{tabular}\\ \\
By the induction hypothesis, $\Gamma\subst{\mtvar}{\basemode''};\cset\subst{\mtvar}{\basemode''} \vdash e_1\subst{\mtvar}{\basemode''}:\classdyn\subst{\mtvar}{\basemode''}$.

Now, consider $\econs = \econsexp{\basemode_{1}}{\mtvar_{1}}{\basemode_{2}}$: $\mtvar_{1}$ must be unique; hence, $(\econsexp{\basemode_{1}}{\mtvar_{1}}{\basemode_{2}}\subst{\mtvar}{\basemode''})$ is $\econsexp{\basemode_{1}\subst{\mtvar}{\basemode''}}{\mtvar_{1}}{\basemode_{2}\subst{\mtvar}{\basemode''}}$ by Lemma \ref{pf:modesubstitution-preserves-submoding}.

Then, by T-Snapshot,
\begin{proofcenter}
$\Gamma\subst{\mtvar}{\basemode''};\cset\subst{\mtvar}{\basemode''}\vdash \snapshot{e_{1}\subst{\mtvar}{\basemode''}}{\basemode_{1}\subst{\mtvar}{\basemode''}}{\basemode_{2}\subst{\mtvar}{\basemode''}}:\texist{\econs\subst{\mtvar}{\basemode''}}.\Cname\lb\mtvar_{1},\listi\subst{\mtvar}{\basemode''}\rb$ \\
\end{proofcenter}

\anote{Come back to prove.}

\end{case}

\begin{case}[\stitlet{MCase}] 
\begin{tabular}[t]{>{$}l<{$} >{$}l<{$} >{$}l<{$}}
e = \mcase{\bt}{e_1} & \t = \tmcase{\bt} & \\
\Gamma;\cset,\basemode\msub\mtvar,\mtvar\msub\basemode'\vdash e_{1_i}:\bt\ \text{for all}\ i & \overline{\moname} = \Fmodes(\programcode) & \\
\end{tabular}\\ \\
Easy.
\end{case}

\begin{case}[\stitlet{ElimCase}] 
\begin{tabular}[t]{>{$}l<{$} >{$}l<{$} >{$}l<{$}}
e = \mcasetag{e_1}{\basemode_1} & \t = \bt & \\
\Gamma;\cset,\basemode\msub\mtvar,\mtvar\msub\basemode' \vdash e_1 : \tmcase{\bt} & \basemode_{1} \in \Fmodes(\programcode)\ \textrm{or}\ \basemode_{1}\ \textrm{appears in}\ \cset,\basemode\msub\mtvar,\mtvar\msub\basemode' &  \\
\end{tabular}\\ \\
Easy.
\end{case}

\begin{case}[\stitlet{Mode}] 
\begin{tabular}[t]{>{$}l<{$} >{$}l<{$} >{$}l<{$}}
e = \moname & \t = \modevt & \\
\end{tabular}\\
Trivial.
\end{case}

\begin{case}[\stitlet{Sub}] 
\begin{tabular}[t]{>{$}l<{$} >{$}l<{$} >{$}l<{$}}
e = e_1 & \t = \t_1' & \\
\Gamma;\cset,\basemode\msub\mtvar,\mtvar\msub\basemode' \vdash e_1:\t_1 & \cset\basemode\msub\mtvar,\mtvar\msub\basemode' \vdash \t_1\tsub\t_1' & \\
\end{tabular}\\ \\
By the induction hypothesis, $\Gamma\subst{\mtvar}{\basemode''};\cset\subst{\mtvar}{\basemode''} \vdash e_1\subst{\mtvar}{\basemode''}:\t_1\subst{\mtvar}{\basemode''}$. Using Lemma \ref{pf:modesubstitution-preserves-subtyping} gives us $\cset\subst{\mtvar}{\basemode''} \vdash \t_1\subst{\mtvar}{\basemode''}\tsub\t_1'\subst{\mtvar}{\basemode''}$

Then, by $\stitlet{Sub}$, we have $\Gamma\subst{\mtvar}{\basemode''};\cset\subst{\mtvar}{\basemode''} \vdash e_1\subst{\mtvar}{\basemode''}:\t'\subst{\mtvar}{\basemode''}$.
\end{case} 

\anote{Finish the proof.}

\end{proof} 

% LEMMA : Term substitution preserves typing
\begin{lemma}[Term Substitution Perserves Typing]
\label{pf:termsubstitution-preserves-typing}
If $\Gamma,\YVAR:\t_0;\cset \vdash e : \t$ and $\Gamma;\cset \vdash \SVAR:\t_0$ then $\Gamma\subst{\YVAR}{\SVAR};\cset \vdash e\subst{\YVAR}{\SVAR}:\t$.
\end{lemma} 

\begin{proof}
Easy induction on the derivation of $\Gamma,\YVAR:\t_{0};\cset \vdash e:\t$.  

\end{proof}

% LEMMA : Relate mtype and mbody
\begin{lemma}
\label{pf:mtype-relates-mbody}
If $\cset\judgewft\classiota$, $\Fmtype(\Mname,\classiota) = \overline{\bt}\rightarrow\bt$ and $\Fmbody(\Mname,\classiota) = \overline{\VAR}.e$ then $\overline{\VAR}:\overline{\bt};\kwthis:\bt;\cset\vdash e:\bt$.
\end{lemma}

\begin{proof}
Induction on the derivation of $\Fmbody(\Mname,\classiota) = \overline{\VAR}.e$ using Lemmas \ref{pf:modesubstitution-preserves-subtyping} and \ref{pf:modesubstitution-preserves-typing}.

\begin{case}[\mbtitlet{Class}] 
\begin{tabular}[t]{>{$}l<{$} >{$}l<{$} >{$}l<{$}}
\overline{\VAR}.e = \overline{\YVAR}.e_0\subst{\listi'}{\listi} & & \\
\kwclass\ \Cname\ \tspec\ \kwextends\ \Dname \{\fdlist\ \mdlist\ \attributor\} & \Feparam(\tspec) = \listi' & \\
\bt_0\ \Mname(\overline{\bt_0}\ \overline{\YVAR}) \{\ e_0\ \} \in \mdlist & & \\
\end{tabular}\\ \\
From \stitlet{Class} and \stitlet{Method} we have $\overline{\YVAR}:\overline{\bt_0};\kwthis:\classiota;\cset'\vdash e_0:\bt_0$. Since $\cset\judgewft\classiota$ we have $\cset\models\cset'\subst{\listi'}{\listi}$ and $\cset' = \Fcons(\tspec)$ from \wftitlet{Class}. Using Lemmas \ref{pf:weakening} and $\ref{pf:modesubstitution-preserves-typing}$ we have $\overline{\YVAR}:\overline{\bt_0}\subst{\listi'}{\listi};\kwthis:\classiota;\cset\vdash e_0\subst{\listi'}{\listi}:\bt_0\subst{\listi'}{\listi}$.

Then, by $\mttitlet{Class}$ we have $\overline{\bt_0}\subst{\listi'}{\listi} = \overline{\bt}$ and $\bt_0\subst{\listi'}{\listi} = \bt$, from which $\overline{\VAR}:\overline{\bt},\kwthis:\classiota;\cset\vdash e:\bt$ is immediate.

\end{case}

\begin{case}[\mbtitlet{Super}] 
\begin{tabular}[t]{>{$}l<{$} >{$}l<{$} >{$}l<{$}}
\overline{\VAR}.e = \Fmbody(\Mname,\Dname\lb\listi\rb) & & \\
\kwclass\ \Cname\ \tspec\ \kwextends\ \Dname \{\fdlist\ \mdlist\ \attributor\} & \Mname \not\in \mdlist & \\
\end{tabular}\\ \\
Immediate from the inductive hypothesis and the fact that $\cset\vdash\classiota\tsub\Dname\lb\listi\rb$.
\end{case} 

\end{proof}

% LEMMA : Relate object and fields
\begin{lemma}
\label{pf:object-fields}
If $\Gamma;\cset\vdash\closure{\alpha}{\classiota}{\overline{\val}}$ and $\Ffields(\classiota) = \overline{\bt}\ \overline{\Fname} = \overline{e}$ then $\Gamma;\cset\vdash\val_i : \bt_i'$.
\end{lemma}

\begin{proof}
Induction on the derivation of $\Ffields(\classiota) = \overline{\t}\ \overline{\Fname}$.

\begin{case}[\fdtitlet{Class}]
\begin{tabular}[t]{>{$}l<{$} >{$}l<{$} >{$}l<{$}}
\kwclass\ \Cname\ \tspec\ \kwextends\ \Dname \{ \overline{\bt_0}\ \overline{\Fname} = \overline{e_0}\ \dots \} & \Feparam(\tspec) = \listi' & \\ 
\Ffields(\Dname\lb\listi\rb) = \overline{\bt_1}\ \overline{\Fname_1} = \overline{e_1} & & \\
\end{tabular}\\ \\
From $\stitlet{Class}$ we have $\varnothing;\cset'\vdash\overline{e_0}:\overline{\bt_0}$. Since $\cset\judgewft\classiota$ we have $\cset\models\cset'\subst{\listi'}{\listi}$ and $\cset' = \Fcons(\tspec)$ from \wftitlet{Class}. Using Lemmas \ref{pf:weakening} and \ref{pf:modesubstitution-preserves-typing} we have $\Gamma;\cset\vdash\overline{e_0}\subst{\listi'}{\listi}:\overline{\bt_0}\subst{\listi'}{\listi}$\\

Now, from T-Object we have $\overline{\bt_1},\overline{\bt_0}\subst{\listi'}{\listi} = \overline{\bt}$ and $\overline{e_1},\overline{e_0}\subst{\listi'}{\listi} = \overline{\val}$. Choosing $\Gamma;\cset\vdash\val_i:\bt_i$ finishes the case.

\end{case}

\begin{case}[\fdtitlet{Object}]
Trivial.
\end{case}

\end{proof} 

\begin{lemma}
\label{pf:fresh-object}
If $\Gamma;\cset\vdash\closure{\alpha}{\classdyn}{\overline{\val}}:\classdyn$ and $\cset\judgewft\Cname\lb\moname,\listi\rb$ then $\Gamma;\cset\vdash\closure{\alpha'}{\Cname\lb\moname,\listi\rb}{\overline{\val}}$.
\end{lemma}

\begin{proof}
\end{proof}


% LEMMA : Preservation (Subject Reduction)
\begin{lemma}[Preservation]
\label{pf:typepreservation}
If $\Gamma;\cset \vdash e:\t$, $e\cloreduct{\moname}e'$, then $\Gamma;\cset\vdash e:\t$.
\end{lemma} 

\begin{proof}
By induction on the derivation of $\Gamma,\cset\vdash e:\t$, with a case analysis on the last rule used.

\begin{case}[\stitlet{Var}]
\begin{tabular}[t]{>{$}l<{$} >{$}l<{$} >{$}l<{$}}
e = \VAR & \t = \Gamma(\VAR) & \\
\end{tabular}\\
Trival: Cannot occur.
\end{case}

\begin{case}[\stitlet{New}] 
\begin{tabular}[t]{>{$}l<{$} >{$}l<{$} >{$}l<{$}}
e = \new{\classiota} & \t = \classiota & \\
\listi = \dynmode,\listi' \textrm{ iff } \kwclass\ \Cname\ \tspec \dots \in \programcode \textrm{ and } \fundef{ethis}(\tspec) = \dynmode & & \\
\listi \neq \dynmode, \listi'  \textrm{ iff } \kwclass\ \Cname\ \tspec \dots \in \programcode \textrm{ and } \fundef{ethis}(\tspec) \neq \dynmode & & \\
\cset \models \Fcons(\tspec) & & \\
\end{tabular}\\
Trivial.
\end{case}

\begin{case}[\stitlet{Cast}] 
\begin{tabular}[t]{>{$}l<{$} >{$}l<{$} >{$}l<{$}}
e = (\bt)e_{1} & \t = \bt & \\
\Gamma;\cset \vdash e_1 : \bt_1 & & \\
\end{tabular}\\

%\begin{subcase-env}

\begin{subcase}
\begin{tabular}[t]{>{$}l<{$} >{$}l<{$} >{$}l<{$}}
e_1 \cloreduct{\moname} e_1' & e' = (\bt)e_1' & \\
\end{tabular}\\
Easy.

\end{subcase}

\begin{subcase}
\begin{tabular}[t]{>{$}l<{$} >{$}l<{$} >{$}l<{$}}
e_1 = \closure{\alpha}{\bt_1}{\overline{\val}} & & \\
(\bt)\closure{\alpha}{\bt_1}{\overline{\val}}\cloreduct{\moname}\closure{\alpha}{\bt_1}{\overline{\val}} & \bt_1\tsub\bt & \\
e' = \closure{\alpha}{\bt_1}{\overline{\val}} & & \\ 
\end{tabular}\\
Trivial. We have $\Gamma;\cset\vdash\closure{\alpha}{\bt_1}{\overline{\val}}:\bt_1$ from T-Cast and T-Obj. Then, by T-Sub we have $\Gamma;\cset\vdash\closure{\alpha}{\bt_1}{\overline{\val}}$.
\end{subcase}

%\end{subcase-env}

\end{case}

\begin{case}[\stitlet{Msg}] 
\begin{tabular}[t]{>{$}l<{$} >{$}l<{$} >{$}l<{$}}
e = e_1.\Mname(\overline{e_1}) & \t = \bt' & \\
\Gamma;\cset \vdash e_{1}:\bt & \Gamma;\cset \vdash \overline{e_{1}}:\overline{\bt} & \Fmtype(\Mname,\bt) = \overline{\bt}\rightarrow\bt' \\ 
\Gamma;\cset \vdash \kwthis:\bt_{this} & \cset \models \{\Fmode(\bt)\msub\Fmode(\bt_{this})\} & \Fmode(\bt) \neq \ ? \\
\end{tabular}\\

%\begin{subcase-env}
\begin{subcase}
\begin{tabular}[t]{>{$}l<{$} >{$}l<{$} >{$}l<{$}}
e_1 \cloreduct{\moname} e_1' & e' = e_1'.(\overline{e_1}) & \\
\end{tabular}\\
Easy.
\end{subcase}

\begin{subcase}
\begin{tabular}[t]{>{$}l<{$} >{$}l<{$} >{$}l<{$}}
e_1 = o & e_{1_i} \cloreduct{\moname} e_{1_i}' & e' = o.(\val_{1_i},\dots,e_{1_i}',\dots,e_n) \\
\end{tabular}\\
Easy.
\end{subcase}

\begin{subcase}[\rtitlet{Msg}]
\begin{tabular}[t]{>{$}l<{$} >{$}l<{$} >{$}l<{$}}
e_1 = o & o = \closure{\alpha}{\Cname\lb\mode,\listi\rb}{\overline{\val}} & \\ 
o.\Mname(\overline{\val'}) \cloreduct{\moname} \redreal{e_b\subst{\overline{\VAR}}{\overline{\val}'}\subst{\kwthis}{\obj}}{\moname'} & \Fmbody(\Mname,\Cname\lb\mode,\listi\rb) = \overline{\VAR}.e_b & \mode\msub\moname \\
\moname' = \Femode(o) & & \\
e' = \redreal{e_b\subst{\overline{\VAR}}{\overline{\val}'}\subst{\kwthis}{\obj}}{\moname'} & & \\
\end{tabular}\\ \\
From Lemma \ref{pf:mtype-relates-mbody} we have $\overline{\VAR}:\overline{\bt},\kwthis:\Cname\lb\mode,\listi\rb;\cset\vdash\overline{\VAR}.e_b:\bt'$. Using Lemma \ref{pf:termsubstitution-preserves-typing} twice gives us $\varnothing;\cset\vdash e_b\subst{\overline{\VAR}}{\overline{\val}'}\subst{\kwthis}{o}:\bt'$.

Now, we may weaken $\varnothing$ to $\Gamma$ by Lemma $\ref{pf:weakening}$ which gives us $\Gamma;\cset\vdash e_b\subst{\overline{\VAR}}{\overline{\val}'}\subst{\kwthis}{o}:\bt'$.

\end{subcase}

%\end{subcase-env}

\end{case}

\begin{case}[\stitlet{Field}] 
\begin{tabular}[t]{>{$}l<{$} >{$}l<{$} >{$}l<{$}}
e = e_1.\Fname_i & \t = \bt_i & \\
\Gamma;\cset \vdash e_{1}:\bt & \Ffields(\bt) = \overline{\bt} \ \overline{\Fname} & \\
\Gamma;\cset \vdash \kwthis:\bt_{this} & \cset \models \{\Fmode(\bt)\msub\Fmode(\bt_{this})\} & \Fmode(\bt) \neq \ ? \\
\end{tabular}\\

%\begin{subcase-env}

\begin{subcase}
\begin{tabular}[t]{>{$}l<{$} >{$}l<{$} >{$}l<{$}}
e_1 \cloreduct{\moname} e_1' & e' = e_1'.\Fname_i & \\
\end{tabular}\\
Easy.
\end{subcase}

\begin{subcase}[\rtitlet{Field}]
\begin{tabular}[t]{>{$}l<{$} >{$}l<{$} >{$}l<{$}}
e_1 = \closure{\alpha}{\Cname\lb\mode,\listi\rb}{\overline{\val}} & & \\
\closure{\alpha}{\Cname\lb\mode,\listi\rb}{\overline{\val}}.\Fname_i \cloreduct{\moname} \val_i & \mode\msub\moname & \\
e' = \val_i & & \\ 
\end{tabular}\\
Lemma \ref{pf:object-fields} gives $\Gamma;\cset\vdash\val_i:\bt_i$ which is exactly what we need.
\end{subcase}

%\end{subcase-env}

\end{case} 

\begin{case}[\stitlet{Snapshot}] 
\begin{tabular}[t]{>{$}l<{$} >{$}l<{$} >{$}l<{$}}
e = \snapshot{e_1}{\basemode_1}{\basemode_2} & \t = \Cname\lb\mtvar,\listi\rb & \\
\Gamma;\cset' \vdash e_1 : \Cname\lb\dynmode,\listi\rb & \mtvar\ fresh & \\
\cset = \cset'\cup\{\basemode_1\msub\mtvar,\mtvar\msub\basemode_2\} & \cset \judgewft \Cname\lb\mtvar,\listi\rb & \\
\end{tabular}\\

%\begin{subcase-env}

\begin{subcase}
\begin{tabular}[t]{>{$}l<{$} >{$}l<{$} >{$}l<{$}}
e_1 \cloreduct{\moname} e_1' & e' = \snapshot{e_1'}{\basemode_1}{\basemode_2} & \\
\end{tabular}\\
Easy.
\end{subcase}

\begin{subcase}[\rtitlet{Snapshot1}]
\begin{tabular}[t]{>{$}l<{$} >{$}l<{$}}
\snapshot{\obj}{\moname_1}{\moname_2} \cloreduct{\moname} \checkp{e_a\subst{\kwthis}{\obj}}{\moname_1}{\moname_2}{\obj} & \obj = \closure{\alpha}{\classdyn}{\overline{\val}} \\ 
\kwclass\ \Cname\ \cdots\ \{\ \cdots\ \attributor \ \} \in \programcode & e_a = \Fabody(\classdyn) \\ 
\end{tabular}\\

From Lemma \ref{pf:abody} we have $\kwthis:\classdyn;\cset'\vdash e_a:\modevt$. Then, by Lemma \ref{pf:termsubstitution-preserves-typing} we have $\varnothing;\cset'\vdash e_a\subst{\kwthis}{\obj}:\modevt$. Using Lemma \ref{pf:weakening} gives us $\Gamma;\cset\vdash e_a\subst{\kwthis}{\obj}:\modevt$.

Now, we have $\cset = \cset'\cup\{\moname_1\msub\mtvar,\mtvar\msub\moname_2\}$ from T-Snapshot, from which $\cset\models\{\moname_1\msub\mtvar,\mtvar\msub\moname_2\}$ is immediate. Then, by T-Check, we have $\Gamma;\cset\vdash\checkp{\subst{\kwthis}{\obj}}{\moname_1}{\moname_2}{\obj}$.

\end{subcase}

\begin{subcase}[\rtitlet{Snapshot2}]
\begin{tabular}[t]{>{$}l<{$} >{$}l<{$} >{$}l<{$}}
\snapshot{\obj}{\basemode_1}{\basemode_2} \cloreduct{\moname} \obj & \obj = \closure{\alpha}{\Cname\lb\moname',\listi\rb}{\overline{\val}} \\ 
\end{tabular}\\
Trivial.
\end{subcase}

%\end{subcase-env}

\end{case}

\begin{case}[\stitlet{MCase}] 
\begin{tabular}[t]{>{$}l<{$} >{$}l<{$} >{$}l<{$}}
e = \mcase{\bt}{e_1} & \t = \tmcase{\bt} & \\
\Gamma;\cset \vdash e_{1_{i}}:\bt \text{ for all} \ i & \overline{\moname} = \Fmodes(\programcode) & \\
\end{tabular}\\

%\begin{subcase-env}

\begin{subcase}
\begin{tabular}[t]{>{$}l<{$} >{$}l<{$} >{$}l<{$}}
e_{1_i} \cloreduct{\moname} e_{1_i}' & e' = \{\moname: \val_{1_i}; \dots; \moname: e_{1_i}'; \dots; \moname: e_{1_n} \} & \\
\end{tabular}\\
Easy.
\end{subcase}

%\end{subcase-env}

\end{case}

\begin{case}[\stitlet{ElimCase}] 
\begin{tabular}[t]{>{$}l<{$} >{$}l<{$} >{$}l<{$}}
e = \mcasetag{e_1}{\basemode} & \t = \bt & \\
\Gamma;\cset \vdash e_{1} : \tmcase{\bt} & \basemode \in \fundef{modes}(\programcode) \textrm{ or } \basemode \textrm{ appears in } \cset &  \\
\end{tabular}\\

%\begin{subcase-env}

\begin{subcase}
\begin{tabular}[t]{>{$}l<{$} >{$}l<{$} >{$}l<{$}}
e_1 \cloreduct{\moname} e_1' & e' = \mcasetag{e_1'}{\basemode} & \\
\end{tabular}\\
Easy.
\end{subcase}

\begin{subcase}[\rtitlet{McaseProj}]
\begin{tabular}[t]{>{$}l<{$} >{$}l<{$} >{$}l<{$}}
e_1 = \mcase{\bt}{\val} & \basemode = \moname_j & \\
\mcasetag{\mcase{\bt}{\val}}{\moname_j} \cloreduct{\moname} \val_j & & \\
e' = \val_j & & \\
\end{tabular}\\
From T-Mcase we have $\overline{\moname} = \Fmodes(\programcode)$ and $\Gamma;\cset\vdash \val_i : \bt\ \text{for all}\ i$. $\Gamma;\cset\vdash \val_j : \bt$ gives us what we need.
\end{subcase}

%\end{subcase-env} 

\end{case}

\begin{case}[\stitlet{ModeValue}] 
\begin{tabular}[t]{>{$}l<{$} >{$}l<{$} >{$}l<{$}}
e = \moname & \t = \modevt \\
\end{tabular}\\
Trival: Cannot occur.
\end{case}

\begin{case}[\stitlet{Sub}] 
\begin{tabular}[t]{>{$}l<{$} >{$}l<{$} >{$}l<{$}}
e = e_{1} & \t = \t_{1}' \\
\Gamma;\cset \vdash e_{1}:\t_{1} & \cset\vdash \t_{1}\tsub\t_{1}' & \\
\end{tabular}\\ \\
Trivial.
\end{case}

\begin{case}[\stitlet{Object}] 
\begin{tabular}[t]{>{$}l<{$} >{$}l<{$} >{$}l<{$}}
e = \closure{\alpha}{\classiota}{\overline{e}} & \t = \classiota \\
\Gamma;\cset \vdash \overline{e}:\overline{\t} & \Ffields(\classiota) = \overline{\t} \ \overline{\Fname} = \overline{e} & \\
\end{tabular}\\ \\
Trival: Cannot occur.
\end{case}

\begin{case}[\stitlet{Check}] 
\begin{tabular}[t]{>{$}l<{$} >{$}l<{$} >{$}l<{$}}
e = \checkp{e_1}{\moname_1}{\moname_2}{e_2} & \t = \Cname\lb\mtvar,\listi\rb & \\
\Gamma;\cset \vdash e_1 : \modevt & \Gamma;\cset \vdash e_2 : \Cname\lb\dynmode,\listi\rb & \\ 
\cset\models\{\basemode_1\msub\mtvar,\mtvar\msub\basemode_2\} & \cset\models\{\basemode_1\msub\moname_1,\moname_2\msub\basemode_2\} & \\
\end{tabular}\\

%\begin{subcase-env}

\begin{subcase}
\begin{tabular}[t]{>{$}l<{$} >{$}l<{$} >{$}l<{$}}
e_1 \cloreduct{\moname} e_1' & e' = \checkp{e_1'}{\moname_1}{\moname_2}{e_2} & \\
\end{tabular}\\
Easy.
\end{subcase}

\begin{subcase}
\begin{tabular}[t]{>{$}l<{$} >{$}l<{$} >{$}l<{$}}
e_2 \cloreduct{\moname} e_2' & e' = \checkp{\moname'}{\moname_1}{\moname_2}{e_2'} & \\
\end{tabular}\\
Easy.
\end{subcase}

\begin{subcase}[\rtitlet{Check}]
\begin{tabular}[t]{>{$}l<{$} >{$}l<{$} >{$}l<{$}}
e_1 = \checkp{\moname'}{\moname_1}{\moname_2}{\obj} & \obj = \closure{\alpha}{\Cname\lb\dynmode,\listi\rb}{\overline{\val}} & \\
\checkp{\moname'}{\moname_1}{\moname_2}{\obj} \cloreduct{\moname} \closure{\alpha'}{\Cname\lb\moname',\listi\rb}{\overline{\val}} & \moname_1\msub\moname'\msub\moname_2 & \\
e' = \closure{\alpha'}{\Cname\lb\moname',\listi\rb}{\overline{\val}} & & \\
\end{tabular}\\

We have $\cset\judgewft\Cname\lb\mtvar,\listi\rb$. Since $\cset\models\{\basemode_1\msub\moname_1,\moname_2\msub\basemode_2\}$ and $\cset\models\{\moname_1\msub\moname',\moname'\msub\basemode_2\}$ we have $\cset\models\{\basemode_1\msub\moname',\moname'\msub\basemode_2\}$; hence, $\cset\judgewft\Cname\lb\moname',\listi\rb$.

From Lemma \ref{pf:fresh-object} we have $\Gamma;\cset\vdash\closure{\alpha'}{\Cname\lb\moname',\listi\rb}{\overline{\val}}:\Cname\lb\moname',\listi\rb$. Then, by T-Sub, $\Gamma;\cset\vdash\closure{\alpha'}{\Cname\lb\moname',\listi\rb}{\overline{\val}}:\Cname\lb\mtvar,\listi\rb$, which is exactly what we need.

\end{subcase} 

%\end{subcase-env}

\end{case}

\begin{case}[\stitlet{Closure}] 
\begin{tabular}[t]{>{$}l<{$} >{$}l<{$} >{$}l<{$}}
e = \cl{\moname'}{e_1} & \t = \t_1 & \\
\Gamma;\cset\vdash e_1 : \t_1 & & \\
\end{tabular}

\begin{subcase}[\rtitlet{Closure1}]
\begin{tabular}[t]{>{$}l<{$} >{$}l<{$} >{$}l<{$}}
e_1 \cloreduct{\moname'} e_1' & e' = \cl{\moname'}{e_1'} & \\
\end{tabular}\\
Trivial.
\end{subcase}

\begin{subcase}[\rtitlet{Closure1}]
\begin{tabular}[t]{>{$}l<{$} >{$}l<{$} >{$}l<{$}}
\cl{\moname'}{\val} \cloreduct{\moname'} \val & e' = \val & \\
\end{tabular}\\
Trivial.
\end{subcase}

\end{case}


\anote{Finish the proof.}

\end{proof}

% LEMMA : Value subtype inversion
\begin{lemma}
\label{pf:value-subtype-inversion}
\leavevmode
\begin{enumerate}[(\arabic*)] 

\item If $\Gamma;\cset\vdash\val:\t$ and $\cset\vdash\t\tsub\classargs{\mode,\overline{\basemode}}$, then $\t = \classargsp{\mode',\overline{\basemode}}$ with $\cset\vdash\classargsp{\mode',\overline{\basemode}}\tsub\classargs{\mode,\overline{\basemode}}$.

\item If $\Gamma;\cset\vdash\val:\t$ and $\cset\vdash\t\tsub\tmcase{\bt}$, then $\t = \tmcase{\bt'}$ with $\cset\vdash\bt'\tsub\bt$.

\end{enumerate}

\end{lemma}

\begin{proof}
\leavevmode

\begin{enumerate}[(\arabic*)] 

\item Case analysis on the induction of the derivation of $\cset\vdash\t\tsub\classargs{\mode,\overline{\basemode}}$: Only S-Dynamic and S-Class apply, we present S-Exists to clarify.

\begin{case}[\sbtitlet{Dynamic}] 
\begin{tabular}[t]{>{$}l<{$} >{$}l<{$} >{$}l<{$}}
\t = \classargs{\mode',\overline{\basemode}} & & \\
\end{tabular}\\ 
Letting $\Cname'$ be $\Cname$ and $\mode$ be $\dynmode$ finishes the case.
\end{case} 

\begin{case}[\sbtitlet{Class}] 
\begin{tabular}[t]{>{$}l<{$} >{$}l<{$} >{$}l<{$}}
\t = \classargsp{\listi} & & \\
%\kwclass\ \Cname'\ \espec\ \kwextends\ \Cname \dots\ \in \programcode & \Feparam(\espec') = \listi' & \cset = \Fcons(\espec) \\
\end{tabular}\\ 
Trivial. Exactly what we need.
\end{case} 

\begin{case}[\sbtitlet{Exists}] 
\begin{tabular}[t]{>{$}l<{$} >{$}l<{$} >{$}l<{$}}
\t = \texist{\econs}.\classargs{\mode,\overline{\basemode}} & & \\ 
\end{tabular}\\ 
If $\t = \texist{\econs}.\classargs{\mode,\overline{\basemode}}$ then we need to have a value with type $\texist{\econs}.\classargs{\mode,\overline{\basemode}}$, but by the structure of our terms and typing rules this cannot occur; hence, S-Exists contradicts our hypothesis and cannot occur.
\end{case} 

\item Induction on the derivation of $\cset\vdash\t\tsub\tmcase{\bt}$: Only S-Mcase applies.

\begin{case}[\sbtitlet{Mcase}] 
\begin{tabular}[t]{>{$}l<{$} >{$}l<{$} >{$}l<{$}}
\t = \tmcase{\bt'} & & \\
\cset\vdash\bt'\tsub\bt & & \\
\end{tabular}\\ 
Trivial. Exactly what we need.
\end{case} 

\end{enumerate}

\end{proof}


% LEMMA : Canonical forms.
\begin{lemma}[Canonical Forms]
\label{pf:canonical-forms}
Given $\Gamma;\cset\vdash\val:\t$,
\leavevmode
\begin{enumerate}[(\arabic*)] 

\item If $\t = \classiota$ then $\val$ has the shape $\closure{\alpha}{\t'}{\overline{\val}}$ with $\cset\vdash\t'\tsub\classiota$.

\item If $\t = \tmcase{\bt}$ then $\val$ has the shape $\mcase{\bt'}{\val}$ with $\cset\vdash\bt'\tsub\bt$.

\item If $\t = \modevt$ then $\val$ has the shape $\moname$ with $\moname \in \Fmodes(\programcode)$.

\end{enumerate}
\end{lemma}

\begin{proof}
\leavevmode

\begin{enumerate}[(\arabic*)] 

\item Induction on the derivation $\Gamma;\cset\vdash\val:\classdyn$. Two rules may apply: T-Obj and T-Sub.

\begin{case}[\stitlet{Obj}] 
\begin{tabular}[t]{>{$}l<{$} >{$}l<{$} >{$}l<{$}}
\val = \closure{\alpha}{\classiota}{\overline{\val}} & & \\ 
\end{tabular}\\ 
Letting $\t'$ be $\classiota$ finishes the case.
\end{case}

\begin{case}[\stitlet{Sub}] 
\begin{tabular}[t]{>{$}l<{$} >{$}l<{$} >{$}l<{$}}
\val = \val_1 & & \\
\Gamma;\cset\vdash\val_1 : \t_1 & \cset\vdash\t_1\tsub\classiota & \\
\end{tabular}\\ 
By Lemma \ref{pf:value-subtype-inversion} $\t_1 = \Cname'\lb\listi\rb$. Then, by the induction hypothesis, $\val_1 = \closure{\alpha}{\t_1'}{\overline{\val}}$ with $\cset\vdash\t_1'\tsub\Cname'\lb\listi\rb$. By S-Trans, $\cset\vdash\t_1'\tsub\classiota$. We may now apply T-Sub to get $\Gamma;\cset\vdash\closure{\alpha}{\t_1'}{\overline{\val}}:\classiota$.
\end{case} 

\item Induction on the derivation $\Gamma;\cset\vdash\val:\tmcase{\bt}$. Two rules may apply: T-Mcase and T-Sub.

\begin{case}[\stitlet{Mcase}] 
\begin{tabular}[t]{>{$}l<{$} >{$}l<{$} >{$}l<{$}}
\val = \mcase{\bt}{\val} & & \\ 
\end{tabular}\\ 
Letting $\bt'$ be $\bt$ finishes the case.
\end{case}

\begin{case}[\stitlet{Sub}] 
\begin{tabular}[t]{>{$}l<{$} >{$}l<{$} >{$}l<{$}}
\val = \val_1 & & \\
\Gamma;\cset\vdash\val_1 : \t_1 & \cset\vdash\t_1\tsub\tmcase{\bt} & \\
\end{tabular}\\ 
\end{case}
By Lemma \ref{pf:value-subtype-inversion} $\t_1 = \tmcase{\bt_1}$ with $\cset\vdash\bt_1\tsub\bt$. Then, by the induction hypothesis, $\val_1 = \mcase{\bt_1'}{\val}$ with $\cset\vdash\bt_1'\tsub\bt_1$. By S-Trans, $\cset\vdash\bt_1'\tsub\bt$. We may now apply T-Sub to get $\Gamma;\cset\vdash\mcase{\bt_1}{\val}:\tmcase{\bt}$.

\item Only T-ModeValue may apply from which $\moname \in \Fmodes(\programcode)$ is immediate.

\end{enumerate} 

\end{proof}

% DEFINITION : Bad Cast
\begin{definition}[Bad Cast]
Expression $(\bt') \closure{\alpha}{\bt}{\overline{\val}}$ is a bad cast iff $\emptyset \vdash \bt \tsub \bt'$ does not hold.
\label{pf:badcast}
\end{definition}

% DEFINITION : Bad Check
\begin{definition}[Bad Check]
Expression $\check{\moname}{\moname'}{\moname''}$ is a bad check iff $\moname' \msub \moname \msub \moname''$ does not hold.
\label{pf:badcheck}
\end{definition} 

% LEMMA : Relate this and the mode of the evaluation context
\begin{lemma}
\label{pf:this-equals-context-mode}
If $\redreal{e}{\moname}$, $\Gamma;\cset\vdash e:\t$ with a premise containing $\Gamma;\cset\vdash\kwthis:\bt_{this}$, then $\Fmode(\bt_{this}) = \moname$.
\end{lemma} 

\begin{proof}
\anote{Come back to prove.}
\end{proof} 

% LEMMA : Progress
\begin{lemma}[Progress]
\label{pf:progress}
If $\Gamma;\cset \vdash e : \t$, then either $e$ is a value, $e$ is a bad cast, $e$ is a bad check, or there exists $e'$ such that $e \cloreduct{\moname} e'$.
\end{lemma}

\begin{proof}
By induction on the derivation of $\Gamma,\cset \vdash e : \t$.

\begin{case}[\stitlet{Var}]
\begin{tabular}[t]{>{$}l<{$} >{$}l<{$} >{$}l<{$}}
e = \VAR & \t = \Gamma(\VAR) & \\
\end{tabular}\\
Trivial.
\end{case}

\begin{case}[\stitlet{New}] 
\begin{tabular}[t]{>{$}l<{$} >{$}l<{$} >{$}l<{$}}
e = \new{\classiota} & \t = \classiota & \\
\end{tabular}\\
Trivial by R-New, with $e' = \closure{\alpha}{\classiota}{\Finit(\programcode,\Cname)}$.
\end{case}

\begin{case}[\stitlet{Cast}] 
\begin{tabular}[t]{>{$}l<{$} >{$}l<{$} >{$}l<{$}}
e = (\bt')e_{1} & \t = \bt' & \\
\Gamma;\cset \vdash e_1 : \classiota & & \\
\end{tabular}\\ \\
By the induction hypothesis, $e_1$ is a value, bad cast, bad check, or there exists $e_1'$ such that $e_1\cloreduct{\moname}e_1'$. 
If $e_1$ is a value, then by Lemma \ref{pf:canonical-forms}, $e_{1} = \closure{\alpha}{\bt}{\overline{\val}}$ with $\cset\vdash\bt\tsub\classiota$. Now, if $\cset\vdash\classiota\tsub\bt'$ then by S-Trans we have $\cset\vdash\bt\tsub\bt'$, from which R-Cast applies, giving $e' = \closure{\alpha}{\bt}{\overline{\val}}$. If $\cset\vdash\classiota\tsub\bt'$ does not hold, then by S-Trans $\cset\vdash\bt\tsub\bt'$ does not hold; hence, we have a bad cast.
If $e_1\cloreduct{\moname}e_1'$ then by the reduction context we may replace $e_1$ with $e_1'$, giving $e' = (\bt')e_{1}'$.
\end{case}

\begin{case}[\stitlet{Msg}] 
\begin{tabular}[t]{>{$}l<{$} >{$}l<{$} >{$}l<{$}}
e = e_{1}.(\overline{e_{1}}) & \t = \bt' & \\
\Gamma;\cset \vdash e_{1}:\bt & \Gamma;\cset \vdash \overline{e_{1}}:\overline{\bt} & \Fmtype(\Mname,\bt) = \overline{\bt}\rightarrow\bt' \\ 
\Gamma;\cset \vdash \kwthis:\bt_{this} & \cset \models \{\Fmode(\bt)\msub\Fmode(\bt_{this})\} & \Fmode(\bt) \neq \ ? \\
\end{tabular}\\ \\
By the induction hypothesis, 
\begin{proofcenter}
$e_1$ is a value, bad cast, bad check, or there exists $e_1'$ such that $e_1\cloreduct{\moname}e_1'$ \\
$e_{1_i}$ is a value, bad cast, bad check, or there exists $e_{1_i}'$ for each $i$ such that $e_{1_i}\cloreduct{\moname}e_{1_i}'$.
\end{proofcenter}
If $e_1\cloreduct{\moname}e_1'$ then we may replace $e_1$ with $e_1'$ giving us $e' = e_{1}'.(\overline{e_{1}})$.

If $e_1$ is a value then by Lemma \ref{pf:canonical-forms}, $e_{1} = \closure{\alpha}{\bt}{\overline{\val}}$ with $\cset\vdash\bt\tsub\classiota$. We consider the case that all $e_{1_i}$ are values first. By Lemma \ref{pf:this-equals-context-mode} we have $\cset \models \{\Fmode(\bt)\msub\moname\}$. R-Msg now applies, giving us $e' = \redreal{e\subst{\overline{\VAR}}{\overline{\val}'}\subst{\kwthis}{\closure{\alpha}{\bt}{\overline{\val}}}}{\moname'}$. Otherwise we may replace the first $e_{1_i}$ with $e_{1_i}'$ giving us $e' = \closure{\alpha}{\bt}{\val_{1_1},\dots,e_{1_i}',\dots,e_{1_n}}$.

\end{case}

\begin{case}[\stitlet{Field}] 
\begin{tabular}[t]{>{$}l<{$} >{$}l<{$} >{$}l<{$}}
e = e_1.\Fname_i & \t = \bt_i & \\
\Gamma;\cset \vdash e_{1}:\bt & \Ffields(\bt) = \overline{\bt} \ \overline{\Fname} & \\
\Gamma;\cset \vdash \kwthis:\bt_{this} & \cset \models \{\Fmode(\bt)\msub\Fmode(\bt_{this})\} & \Fmode(\bt) \neq \ ? \\
\end{tabular}\\ \\
By the induction hypothesis, $e_1$ is a value, bad cast, bad check, or there exists $e_1'$ such that $e_1\cloreduct{\moname}e_1'$. 

If $e_1\cloreduct{\moname}e_1'$ then we may replace $e_1$ with $e_1'$ giving us $e' = e_1'.\Fname_i$. If $e_1$ is a value then by Lemma \ref{pf:canonical-forms}, $e_{1} = \closure{\alpha}{\bt}{\overline{\val}}$ with $\cset\vdash\bt\tsub\classiota$. By Lemma \ref{pf:this-equals-context-mode} we have $\cset \models \{\Fmode(\bt)\msub\moname\}$. R-Field now applies, giving us $e' = \val_i$.
\end{case} 

\begin{case}[\stitlet{Snapshot}] 
\begin{tabular}[t]{>{$}l<{$} >{$}l<{$} >{$}l<{$}}
e = \snapshot{e_{1}}{\basemode_{1}}{\basemode_{2}} & \t = \texist{\econs}. \Cname\lb\mtvar,\listi\rb &  \\
\Gamma;\cset \vdash e_{1} : \Cname\lb?,\listi\rb & \econs = \basemode_1 \msub \mtvar \msub \basemode_2 & \\
\end{tabular}\\ \\
By the induction hypothesis, $e_{1}$ is a value, bad cast, bad check, or there exists $e_{1}'$ such that $e_{1}\cloreduct{\moname}e_{1}'$. 

If $e_{1}$ is a value then by Lemma \ref{pf:canonical-forms} $e_{1} = \closure{\alpha}{\bt}{\overline{\val}}$ with $\cset\vdash\bt\tsub\Cname\lb\dynmode,\listi\rb$. Now, if $\Feargs(\bt) = \dynmode,\listi$, then R-Snapshot1 applies, with $e' = \letx{\check{ \attributor \subst{\kwthis}{\closure{\alpha}{\bt}{\overline{\val}}}}{\moname'}{\moname''}}{\closure{\alpha'}{\bt\subst{\Feargs(\bt)}{\VAR,\listi}}{\overline{\val}}}$. Otherwise $\Feargs(\bt) = \moname,\listi$ from which we may apply R-Snapshot2 to get $e' = \letx{\check{\moname}{\moname'}{\moname''}}{\closure{\alpha}{\bt}{\overline{\val}}}$.

If $e_{1}\cloreduct{\moname}e_{1}'$ then by the reduction context we may replace $e_{1}$ with $e_{1}'$ to get $e' = \snapshot{e_{1}'}{\basemode_{1}}{\basemode_{2}}$.

\end{case}

\begin{case}[\stitlet{MCase}] 
\begin{tabular}[t]{>{$}l<{$} >{$}l<{$} >{$}l<{$}}
e = \mcase{\bt}{e_{1}} & \t = \tmcase{\bt} & \\
\Gamma;\cset \vdash e_{1_{i}}:\bt \text{ for all} \ i & \overline{\moname} = \Fmodes(\programcode) & \\
\end{tabular}\\ \\
By the induction hypothesis, $e_{1_{i}}$ is a value, bad cast, bad check, or there exists $e_{1_{i}}'$ such that $e_{1_{i}}\cloreduct{\moname}e_{1_{i}}'$. 

If all $e_{1_{i}}$ are values, then $e$ is a value. Otherwise by the reduction context we may replace $e_{1_{i}}$ with $e_{1_{i}}'$, giving us $e'$.
\end{case}

\begin{case}[\stitlet{ElimCase}] 
\begin{tabular}[t]{>{$}l<{$} >{$}l<{$} >{$}l<{$}}
e = \mcasetag{e_{1}}{\basemode} & \t = \bt & \\
\Gamma;\cset \vdash e_{1} : \tmcase{\bt} & \basemode \in \fundef{modes}(\programcode) \textrm{ or } \basemode \textrm{ appears in } \cset &  \\
\end{tabular}\\ \\
By the induction hypothesis, $e_{1}$ is a value, bad cast, bad check, or there exists $e_{1}'$ such that $e_{1}\cloreduct{\moname}e_{1}'$. 

If $e_{1}$ is a value then by Lemma \ref{pf:canonical-forms}, $e_{1}$ has the shape $\mcase{\bt}{\val}$, from which R-McaseProj applies, giving us $e' = \val_{j}$. 
If $e_{1}\cloreduct{\moname}e_{1}'$ then by the reduction context we may replace $e_{1}$ with $e_{1}'$, giving us $e' = \mcasetag{e_{1}'}{\basemode}$.

\end{case}

\begin{case}[\stitlet{Mode}] 
\begin{tabular}[t]{>{$}l<{$} >{$}l<{$} >{$}l<{$}}
e = \moname & \t = \modevt \\
\end{tabular}\\
Trivial.
\end{case}

\begin{case}[\stitlet{Sub}] 
\begin{tabular}[t]{>{$}l<{$} >{$}l<{$} >{$}l<{$}}
e = e_{1} & \t = \t_{1}' \\
\Gamma;\cset \vdash e_{1}:\t_{1} & \cset\vdash \t_{1}\tsub\t_{1}' & \\
\end{tabular}\\ \\
By the induction hypothesis, $e_{1}$ is a value, bad cast, bad check, or there exists $e_{1}'$ such that $e_{1}\cloreduct{\moname}e_{1}'$. 

If $e_{1}$ is a value, we are done. If $e_{1}\cloreduct{\moname}e_{1}'$ then we may replace $e_{1}$ with $e_{1}'$ giving us $e' = e_{1}'$.
\end{case}

\begin{case}[\stitlet{Object}] 
\begin{tabular}[t]{>{$}l<{$} >{$}l<{$} >{$}l<{$}}
e = \closure{\alpha}{\classiota}{\overline{e}} & \t = \classiota \\
\Gamma;\cset \vdash \overline{e}:\overline{\t} & \Ffields(\classiota) = \overline{\t} \ \overline{\Fname} = \overline{e} & \\
\end{tabular}\\ \\
By the induction hypothesis, $e_{i}$ is a value, bad cast, bad check, or there exists $e_{i}'$ such that $e_{i}\cloreduct{\moname}e_{i}'$ for each $i$. 

If all $e_{i}$ are values, then $e$ is a value and we are done. Otherwise, by the reduction context, we may replace an $e_{i}$ with $e_{i}'$, giving us $e' = \closure{\alpha}{\classiota}{\val_{1},\dots,e_{i}',\dots,e_{n}}$.
\end{case}

\begin{case}[\stitlet{Check}] 
\begin{tabular}[t]{>{$}l<{$} >{$}l<{$} >{$}l<{$}}
e = \check{e_1}{\moname_1}{\moname_2} & \t = \modevt & \\
\Gamma;\cset \vdash e_1 : \modevt \\
\end{tabular}\\ \\
By the induction hypothesis, $e_1$ is a value, bad cast, bad check, or there exists $e_1'$ such that $e_1\cloreduct{\moname}e_1'$.

If $e_1$ is a value, then by Lemma \ref{pf:canonical-forms}, $e_1$ has the shape $\moname$. Now, we have two cases: If $\moname_1\msub\moname\msub\moname_2$ then R-Check applies, giving us $e' = \moname$. If $\moname_1\msub\moname\msub\moname_2$ \emph{does not hold} then by definition we have a bad check.

If $e_1\cloreduct{\moname}e_1'$ then we may replace $e_1$ with $e_1'$ by the reduction context, giving us $e' = \check{e_1'}{\moname_1}{\moname_2}$.

\end{case}

\end{proof}



% THEOREM : Type soundness
\begin{theorem}[Type Soundness]
\label{pf:staticsoundness}
If $ \programcode$ is well-typed and $\Fboot(\programcode) = \lb\top,e\rb$, then either $e \cloreduct{\top}_{*} \val$, $\lb\top,e\rb \Uparrow$, or $e \cloreduct{\top}_{*} e'$ and $e'$ is a bad cast or a bad check.
\end{theorem}

Let us say $\lb\moname_0;e_0\rb$ is a \emph{sub-redex} of reduction $e \cloreduct{\moname} e'$ iff $e_0 \cloreduct{\moname_0} e'_0$ is a sub-derivation of $e \cloreduct{\moname} e'$. We next state two important properties of \ourlang{}. 

% THEOREM : Type decidability
\begin{theorem}[Type Decidability]
\label{pf:typedecidability}
For any program $\programcode$, it is decidable whether $\vdash \programcode$ holds.
\end{theorem} 

% THEOREM : Monotone snapshotting
\begin{theorem}[Monotone Snapshotting]

If $ \programcode$ is well-typed, $\Fboot(\programcode) = \lb\top,e\rb$, $e \cloreduct{\top} \dots e_1 \cloreduct{\top} e_2 \dots\cloreduct{\top} e_3 \cloreduct{\top} e_4$,  $\lb\moname; \closure{ \alpha, \bt}{\overline{\val}} \rb$ is a sub-redex of $e_1 \cloreduct{\top} e_2$ and $\lb\moname'; \closure{ \alpha, \bt'}{\overline{\val}'} \rb$ is a sub-redex of $e_3 \cloreduct{\top} e_4$, then if $\fundef{mode}(\bt) \neq \dynmode$, $\bt = \bt'$. 
\end{theorem}

In other words, once the type of an object becomes static, it can never be changed any more. This theorem reveals the \emph{monotone} nature of object type change throughout the object lifetime, a crucial property to guarantee type soundness. 

% THEOREM : Waterfall invariant with hybrid typing.
\begin{theorem}[Waterfall Invariant with Hybrid Typing]
\label{pf:waterfallinvariant}

If $ \programcode$ is well-typed, $\Fboot(\programcode) = \lb\top,e\rb$, $e \cloreduct{\top} \dots e_1 \cloreduct{\top} e_2$, and $\lb\moname,\closure{\alpha, \bt}{\overline{\val}}.\Mname(\overline{\val'})\rb$ or $\lb\moname,\closure{\alpha, \bt}{\overline{\val}}.\Fname(\overline{\val'})\rb$ is a sub-redex of $e_1 \cloreduct{\top} e_2$, then $\rodef \models \Fmode(\bt) \tsub \moname$ where $\programcode = \rodef \ \overline{\classes} \ e$.
\end{theorem} 

This theorem says even in the presence of hybrid typing, waterfall invariant --- a key principle to regulate mode-based energy management --- is still preserved. Observe that this theorem says run-time errors are never delayed to messaging or field access time. If any potential violation may happen due to dynamic typing, a run-time error would result from a bad check, \emph{i.e.}, at snapshotting time.

\end{document}


mm
